%!TEX root = skripsi.tex
%-----------------------------------------------------------------------------%
\chapter*{\kataPengantar}
%-----------------------------------------------------------------------------%

\begin{arabtex}الْحَمْدُ لِلَّـهِ الَّذِي هَدَانَا لِهَـٰذَا وَمَا كُنَّا لِنَهْتَدِيَ لَوْلَا أَنْ هَدَانَا اللَّـهُ\end{arabtex} \f{Segala puji bagi Allah yang telah menunjuki kami kepada (surga) ini. Dan kami sekali-kali tidak akan mendapat petunjuk kalau Allah tidak memberi kami petunjuk. [Al-A'raf:43]}\begin{arabtex}الْحَمْدُ للهِ رَبِّ الْعَالَمِيْنَ، وَالصَّلاَةُ وَالسَّلاَمُ عَلَى سَيِّدِنَا مُحَمَّدٍ سَيِّدِ اْلأَوَّلِيْنَ وَالآخِرِيْنَ، وَعَلَى آلِهِ وَصَحْبِهِ وَمَنِ اهْتَدَى بِهَدْيِهِ إِلَى يَوْمِ الدَّيْنِ\end{arabtex} \f{Segala puji bagi Allah, Tuhan sekalian alam, semoga keselamatan dan kesejahteraan tetap terlimpahkan atas junjungan kita Nabi Muhammad SAW, penghulu manusia, baik yang dahulu maupun yang belakangan, begitu juga kepada segenap keluarga dan semua orang yang mengikuti petunjuk, sampai saat Hari Kemudian.} Segala puji dan syukur kehadirat Allah SWT, Tuhan Yang Maha Esa, yang senantiasa memberikan ramhat dan hidayah-Nya, sehingga \saya~dapat menyelesaikan skripsi ini.

Penulisan skripsi ini ditujukan untuk memenuhi salah satu syarat untuk menyelesaikan pendidikan pada Program \gelar, Universitas Indonesia. \Saya~sadar bahwa dalam perjalanan menuntut ilmu di universitas hingga dalam menyelesaikan skripsi ini, \saya~tidak sendiri. \Saya~ingin berterima kasih kepada pihak-pihak yang selalu peduli, mendampingi, dan mendukung \saya, yaitu:

\begin{enumerate}
	\item Kedua Orang Tua \saya~yang selalu memberikan dukungan dan do'a kepada \saya.
	\item Dra. Mirna Adriani, Ph.D. dan Dr. Amalia Zahra selaku dosen pembimbing yang banyak memberikan arahan, masukan, dan bantuan dalam menyelesaikan skripsi ini.
  \item Alfan Farizki Wicaksono, ST., M.Sc. dan Rahmad Mahendra, S.Kom., M.Sc. yang memberi dukungan dari awal sampai akhir pengerjaan skripsi ini, dan juga memberikan tips-tips dalam mengerjakan skripsi.
  \item Andreas Febrian yang telah membuat \f{template} dokumen skripsi ini, sehingga \saya~menjadi terbantu dalam menulis skripsi.
  \item Erik Dominikus yang telah mempublikasikan dan mempopulerkan \f{template} dokumen skripsi ini, sehingga \saya~menjadi tahu bahwa ada \f{template} tersebut.
	\item Mohammad Syahid Wildan dan Abid Nurul Hakim, sebagai rekan yang banyak memberi masukan dan berbagi ide dengan \saya.
  \item Teman-teman Lab Information Retrieval yang memberi dukungan dan semangat kepada \saya~untuk menyelesaikan skripsi ini.
  \item Teman-teman Forum Remaja Masjid UI yang memberi dukungan serta do'a kepada \saya~untuk menyelesaikan skripsi ini.
	\item Pihak-pihak lain yang tidak dapat \saya~sebutkan satu-persatu yang sudah memberikan bantuan dan dukungannya kepada \saya.
\end{enumerate}
\vspace*{0.1cm}
\begin{flushright}
Depok, Juni 2016\\[0.1cm]
\vspace*{1cm}
\penulis

\end{flushright}