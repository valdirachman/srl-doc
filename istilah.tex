%!TEX root = skripsi.tex
%
% @author  Andreas Febrian
% @version 1.00
% 
% Mendaftar seluruh istilah yang mungkin akan perlu dijadikan 
% italic atau bold pada setiap kemunculannya dalam dokumen. 
% 

\var{\license}{\f{Creative Common License 1.0 Generic}}
\var{\bslash}{$\setminus$}
\var{\ioa}{\f{input}}
\var{\iob}{\f{output}}
\var{\br}{\f{bit rate}}
\var{\fr}{\f{frame}}
\var{\quran}{Al-Qur'an}
\var{\mer}{MER}
\var{\rnn}{\f{Recurrent Neural Network}}
\var{\ol}{\f{online}}
\var{\disease}{\f{disease}}
\var{\symptom}{\f{symptom}}
\var{\drug}{\f{drug}}
\var{\treatment}{\f{treatment}}
\var{\Disease}{\f{Disease}}
\var{\Symptom}{\f{Symptom}}
\var{\Drug}{\f{Drug}}
\var{\Treatment}{\f{Treatment}}
\var{\we}{\f{word embedding}}
\var{\radit}{Raditya Herwando}
\var{\skripsiRadit}{\cite{skripsiKakRadit}}