%!TEX root = skripsi.tex
%-----------------------------------------------------------------------------%
% Informasi Mengenai Dokumen
%-----------------------------------------------------------------------------%
% 
% Judul laporan. 
\var{\judul}{Semantic Role Labeling in Indonesian Conversational Language using Recurrent Neural Networks}
% 
% Tulis kembali judul laporan, kali ini akan diubah menjadi huruf kapital
\Var{\Judul}{Semantic Role Labeling in Indonesian Conversational Language using Recurrent Neural Networks}
% 
% Tulis kembali judul laporan namun dengan bahasa Ingris
\var{\judulInggris}{Semantic Role Labeling in Indonesian Conversational Language using Recurrent Neural Networks}

% 
% Tipe laporan, dapat berisi Skripsi, Tugas Akhir, Thesis, atau Disertasi
\var{\type}{Skripsi}
% 
% Tulis kembali tipe laporan, kali ini akan diubah menjadi huruf kapital
\Var{\Type}{Skripsi}
% 
% Tulis nama penulis 
\var{\penulis}{Valdi Rachman}
% 
% Tulis kembali nama penulis, kali ini akan diubah menjadi huruf kapital
\Var{\Penulis}{Valdi Rachman}
% 
% Tulis NPM penulis
\var{\npm}{1306381862}
% 
% Tuliskan Fakultas dimana penulis berada
\Var{\Fakultas}{Ilmu Komputer}
\var{\fakultas}{Ilmu Komputer}
% 
% Tuliskan Program Studi yang diambil penulis
\Var{\Program}{Ilmu Komputer}
\var{\program}{Ilmu Komputer}
\var{\programEng}{Computer Science}
% 
% Tuliskan tahun publikasi laporan
\Var{\bulanTahun}{June 2017}
% 
% Tuliskan gelar yang akan diperoleh dengan menyerahkan laporan ini
\var{\gelar}{Sarjana Ilmu Komputer}
% 
% Tuliskan tanggal pengesahan laporan, waktu dimana laporan diserahkan ke 
% penguji/sekretariat
\var{\tanggalPengesahan}{13 Januari 2017}
% 
% Tuliskan tanggal keputusan sidang dikeluarkan dan penulis dinyatakan 
% lulus/tidak lulus
\var{\tanggalLulus}{-}
% 
% Tuliskan pembimbing 
\var{\pembimbing}{Rahmad Mahendra}
\var{\pembimbingdua}{Alfan Farizki Wicaksono}
% 
% Alias untuk memudahkan alur penulisan paa saat menulis laporan
\var{\saya}{we}
\var{\Saya}{We}

%-----------------------------------------------------------------------------%
% Judul Setiap Bab
%-----------------------------------------------------------------------------%
% 
% Berikut ada judul-judul setiap bab. 
% Silahkan diubah sesuai dengan kebutuhan. 
% 
\Var{\kataPengantar}{Remarks}
\Var{\babSatu}{Introduction}
\Var{\babDua}{Literature Review}
\Var{\babTiga}{Methodology}
\Var{\babEmpat}{Implementation}
\Var{\babLima}{Experiments}
\Var{\babEnam}{Conclusions}
