%!TEX root = skripsi.tex
\section{KAMUS DISEASE}
\begin{longtable}{|p{.25\textwidth}|p{.25\textwidth}|p{.25\textwidth}|p{.25\textwidth}|}

	\caption{Daftar Kata dalam Kamus \textit{Disease}}\label{lampiran:disease}\\
	% header and footer information
	\hline
	Spina bifida & Fenilketonuria & Duchene muscular dystrophy & Kejang demam \\ \hline
	Infeksi sitomegalovirus & Meningitis & Ensefalitis & Malaria serebral \\ \hline
	Tetanus & Tetanus neonato rum & Toksoplasmosis serebral & Abses otak \\ \hline
	HIV AIDS & AIDS & Hidrosefalus & Poliomielitis \\ \hline
	Rabies & Spondilitis TB & Tumor primer & Tumor sekunder \\ \hline
	Ensefalopati & Koma & Mati batang otak & Tension headache \\ \hline
	Migren & Arteritis kranial & Neuralgia trigeminal & Cluster headache \\ \hline
	TIA & Infark serebral & Hematom intraserebral & Perdarahan subarakhnoid \\ \hline
	Ensefalopati hipertensi & BellsâA ? Z palsy ´ & Lesi batang otak & Meniere’s disease \\ \hline
	Vertigo & Benign paroxysmal positional vertigo & Cerebral palsy & Demensia \\ \hline
	Alzheimer & Gangguan Pergerakan & Parkinson & Gangguan pergerakan lainnya \\ \hline
	Kejang & Epilepsi & Status epileptikus & Sklerosis multipel \\ \hline
	Amyotrophic lateral sclerosis & ALS & Complete spinal transaction & Sindrom kauda equine \\ \hline
	Neurogenic bladder & Siringomielia & Mielopati & Dorsal root syndrome \\ \hline
	Acute medulla compression & Radicular syndrome & Hernia nucleus pulposus & HNP \\ \hline
	Hematom epidural & Hematom subdural & Trauma Medula Spinalis & Reffered pain \\ \hline
	Nyeri neuropatik & Sindrom Horner & Carpal tunnel syndrome & Tarsal tunnel syndrome \\ \hline
	Neuropati & Peroneal palsy & Guillain Barre syndrome & Miastenia gravis \\ \hline
	Polimiositis & Neurofibromatosis & Von Recklaing Hausen disease & Amnesia \\ \hline
	Afasia & Mild Cognitive Impairment & MCI & Skizofrenia \\ \hline
	Gangguan waham & Gangguan psikotik & Gangguan skizoafektif & Gangguan bipolar \\ \hline
	Gangguan siklotimia & Depresi endogen & Gangguan distimia & depresi neurosis \\ \hline
	Gangguan depresif & Baby blues & post-partum depression & Agorafobia dengan/tanpa panik \\ \hline
	Fobia sosial & Fobia spesifik & Gangguan panik & Gangguan cemas menyeluruh \\ \hline
	Gangguan campuran cemas depresi & Gangguan obsesif kompulsif & Reaksi terhadap stres yg berat & gangguan penyesuaian \\ \hline
	Post traumatic stress disorder & Gangguan disosiasi & Gangguan somatoform & Trikotilomania \\ \hline
	Gangguan kepribadian & Gangguan identitas gender & Gangguan preferensi seksual & Gangguan perkembangan pervasif \\ \hline
	Retardasi mental & Gangguan pemusatan perhatian dan hiperaktif & autisme & Gangguan tingkah laku \\ \hline
	conduct disorder & Anoreksia nervosa & Bulimia & Pica \\ \hline
	Gilles de la tourette syndrome & Chronic motor of vocal tics disorder & Transient tics disorder & Functional encoperasis \\ \hline
	Functional enuresis & Uncoordinated speech & Parafilia & Gangguan keinginan dan gairah seksual \\ \hline
	Gangguan orgasmus & Sexual pain disorder & Insomnia & Hipersomnia \\ \hline
	Sleep-wake cycle disturbance & Nightmare & Sleep walking & Benda asing di konjungtiva \\ \hline
	Konjungtivitis & Pterigium & Perdarahan subkon jungtiva & Mata kering \\ \hline
	Blefaritis & Hordeolum & Chalazion & Laserasi kelopak mata \\ \hline
	Entropion & Trikiasis & Lagoftalmus & Epikantus \\ \hline
	Ptosis & Retraksi kelopak mata & Xanthelasma & Dakrioadenitis \\ \hline
	Dakriosistitis & Dakriostenosis & Laserasi duktus lakrima & Skleritis \\ \hline
	Episkleritis & Erosi & Benda asing di kornea & Luka bakar kornea \\ \hline
	Keratitis & Kerato konjungtivitis sicca & Edema kornea & Keratokonus \\ \hline
	Xerophtalmia & Endoftalmitis & Mikroftalmos & Hifema \\ \hline
	Hipopion & Perdarahan Vitreous & Iridosisklitis & iritis \\ \hline
	Tumor iris & Katarak & Afakia kongenital & Dislokasi lensa \\ \hline
	Hipermetropia ringan & Miopia ringan & Astigmatism ringan & Presbiopia \\ \hline
	Anisometropia & Ambliopia & Diplopia binokuler & Buta senja \\ \hline
	Skotoma & Hemianopia & bitemporal & homonymous \\ \hline
	Gangguan lapang pandang & Ablasio retina & Perdarahan retina & oklusi pembuluh darah retina \\ \hline
	Degenerasi makula & Retinopati & Korioretinitis & Optic disc cupping \\ \hline
	Edema papil & Atrofi optik & Neuropati optik & Neuritis optik \\ \hline
	Glaukoma akut & Tuli kongenital & Tuli perseptif & Tuli konduktif \\ \hline
	Inflamasi pada aurikular & Herpes zoster pada telinga & Fistula preaurikular & Labirintitis \\ \hline
	Otitis eksterna & Otitis media akut & Otitis media serosa & Otitis media kronik \\ \hline
	Mastoiditis & Miringitis bullosa & Perforasi membran timpani & Otosklerosis \\ \hline
	Timpanosklerosis & Kolesteatoma & Presbiakusis & Serumen prop \\ \hline
	Mabuk perjalanan & Trauma akustik akut & Trauma aurikular & Deviasi septum hidung \\ \hline
	Furunkel pada hidung & Rhinitis akut & Rhinitis vasomotor & Rhinitis alergika \\ \hline
	Rhinitis kronik & Rhinitis medika mentosa & Sinusitis & Sinusitis frontal akut \\ \hline
	Sinusitis maksilaris akut & Sinusitis kronik & Benda asing & Epistaksis \\ \hline
	Etmoiditis akut & Polip & Fistula dan kistabrankial lateral danmedial & Higroma kistik \\ \hline
	Tortikolis & Abses Bezold & Influenza & Pertusis \\ \hline
	Acute Respiratorydistress syndrome & ARDS & SARS & Flu burung \\ \hline
	Faringitis & Tonsilitis & Laringitis & Hipertrofi adenoid \\ \hline
	Abses peritonsilar & Pseudo-croop acute epiglotitis & Difteria & Karsinoma laring \\ \hline
	Karsinoma nasofaring & Trakeitis & Aspirasi & Asma bronkial \\ \hline
	Status asmatikus & asma akut berat & Bronkitis akut & Bronkiolitis akut \\ \hline
	Bronkiektasis & Displasia bronkopulmonar & Karsinoma paru & Pneumonia, bronkopneumonia \\ \hline
	Pneumonia aspirasi & Tuberkulosis paru tanpa komplikasi & Tuberkulosis dengan HIV & Multi Drug Resistance \\ \hline
	MDR & Pneumothorax ventil & Pneumothorax & Efusi pleura \\ \hline
	Efusi pleura masif & Emfisema paru & Atelektasis & Paru Obstruksi Kronik (PPOK) eksaserbasi akut \\ \hline
	Edema paru & Infark paru & Abses paru & Emboli paru \\ \hline
	Kistik fibrosis & Haematothorax & Tumor mediastinum & Pnemokoniasis \\ \hline
	Penyakit paru intersisial & Obstructive Sleep Apnea & OSA & Kelainan jantung congenital \\ \hline
	Ventricular Septal Defect & Atrial Septal Defect & Patent Ductus Arteriosus & Tetralogy of Fallot \\ \hline
	Radang pada dinding jantung & Endokarditis & Miokarditis & Perikarditis \\ \hline
	Syok & septik & hipovolemik & kardiogenik \\ \hline
	neurogenik & Angina pektoris & Infark miokard & Gagal jantung akut \\ \hline
	Gagal jantung kronik & Cardio respiratory arrest & Kelainan katup jantung & Mitral stenosis \\ \hline
	Mitral regurgitation & Aortic stenosis & Aortic regurgitation & Takikardi \\ \hline
	supraventrikular & ventrikular & Fibrilasi atrial & Fibrilasi ventrikular \\ \hline
	Atrial flutter & Ekstrasistol supraventrikular & Ekstrasistol ventrikular & Bundle Branch Block \\ \hline
	Aritmia lainnya & Kardiomiopati & Kor pulmonale akut & Kor pulmonale kronik \\ \hline
	Hipertensi esensial & Hipertensi sekunder & Hipertensi pulmoner & Raynaud \\ \hline
	Trombosis arteri & Koarktasio aorta & Buerger’s (Thromboangiitis Obliterans) & Emboli arteri \\ \hline
	Aterosklerosis & Subclavian stealsyndrome & Aneurisma Aorta & Aneurisma diseksi \\ \hline
	Klaudikasio & jantung reumatik & Tromboflebitis & Limfangitis \\ \hline
	Varises & Varises primer & Varises sekunder & Obstructed venousreturn \\ \hline
	Trombosis vena dalam & Emboli vena & Limfedema (primer, sekunder) & Insufisiensi venakronik \\ \hline
	Sumbing pada bibir & Sumbing pada palatum & Micrognatia & macrognatia \\ \hline
	Kandidiasis mulut & Ulkus mulut & Glositis & Leukoplakia \\ \hline
	Angina Ludwig & Parotitis & Karies gigi & Atresia esofagus \\ \hline
	Akalasia & Esofagitis refluks & Lesi korosif pada esofagus & Varises esofagus \\ \hline
	Ruptur esofagus & Hernia & Hernia diaframatika & Hernia hiatus \\ \hline
	Hernia umbilikalis & Peritonitis & Perforasi usus & Malrotasi traktusgastro-intestinal \\ \hline
	Infeksi pada umbilikus & Sindrom Reye & Gastritis & Gastroenteritis \\ \hline
	kolera & giardiasis & Refluks gastroesofagus & Ulkus (gaster, duodenum) \\ \hline
	Stenosis pilorik & Atresia intestinal & Divertikulum Meckel & Fistula umbilikal, omphalos cole gastroschisis \\ \hline
	Apendisitis akut & Abses apendiks & Demam tifoid & Perdarahan gastrointestinal \\ \hline
	Ileus & Malabsorbsi & Intoleransi makanan & Alergi makanan \\ \hline
	Keracunan makanan & Botulisme & Penyakit cacing tambang & Strongiloidiasis \\ \hline
	Askariasis & Skistosomiasis & Taeniasis & Pes \\ \hline
	Hepatitis A & Hepatitis B & Hepatitis C & Abses heparamoeba \\ \hline
	Perlemakan hepar & Sirosis hepatis & Gagal hepar & Neoplasma hepar \\ \hline
	Kolesistitis & Kole(doko)litiasis & Empiema & hidrops kandung empedu \\ \hline
	Atresia biliaris & Pankreatitis & Karsinoma pankreas & Divertikulosis \\ \hline
	divertikulitis & Kolitis & Disentri basiler, disentri amuba & Penyakit Crohn \\ \hline
	Kolitis ulseratif & Irritable Bowel Syndrome & Polip/adenoma & Karsinoma kolon \\ \hline
	Penyakit Hirschsprung & Enterokolitis nekrotik & Intususepsi & invaginasi \\ \hline
	Atresia anus & Proktitis & Abses anal & Hemoroid grade 1 \\ \hline
	Hemoroid grade 2 & Hemoroid grade 3 & Hemoroid grade 4 & Fistula \\ \hline
	Fisura anus & Prolaps rektum & Prolaps anus & Limfoma \\ \hline
	Gastrointestinal Stromal Tumor & GIST & Infeksi saluran kemih & Glomerulonefritis Akut \\ \hline
	Glomerulonefritis Kronik & Gonore & Karsinoma sel renal & Tumor Wilms \\ \hline
	Acute kidney injury & Penyakit ginjal kronik & Sindrom nefrotik & Kolik renal \\ \hline
	Batu saluran kemih & Ginjal polikistik simtomatik & Ginjal tapal kuda & Pielonefritis tanpa komplikasi \\ \hline
	Nekrosis tubular akut & Hipospadia & Epispadia & Testis tidak turun \\ \hline
	kriptorkidismus & Rectratile testis & Varikokel & Hidrokel \\ \hline
	Fimosis & Parafimosis & Spermatokel & Epididimitis \\ \hline
	Prostatitis & Torsio testis & Ruptur uretra & Ruptur kandung kencing \\ \hline
	Ruptur ginjal & Karsinoma uroterial & Seminoma testis & Teratoma testis \\ \hline
	Hiperplasia prostat jinak & Karsinoma prostat & Striktura uretra & Priapismus \\ \hline
	Chancroid & Sifilis & Toksoplasmosis & Sindrom duh genital \\ \hline
	Infeksi virus Herpestipe 2 & Infeksi saluran kemih bagian bawah & Vulvitis & Kondiloma akuminatum \\ \hline
	Vaginitis & Vaginosis bakterialis & Servisitis & Salpingitis \\ \hline
	Abses tubo ovarium & radang panggul & Infeksi intra-uterin & korioamnionitis \\ \hline
	TORCH & malaria & Aborsi mengancam & Aborsi spontaninkomplit \\ \hline
	Aborsi spontankomplit & Hiperemesis gravidarum & Inkompatibilitas darah & Mola hidatidosa \\ \hline
	Hipertensi pada kehamilan & Preeklampsia & Eklampsia & Diabetes gestasional \\ \hline
	Kehamilan posterm & Insufisiensi plasenta & Plasenta previa & Vasa previa \\ \hline
	Abrupsio plasenta & Inkompeten serviks & Polihidramnion & Kelainan letak janin setelah 36 minggu \\ \hline
	Kehamilan ganda & Janin tumbuh lambat & Kelainan janin & Diproporsi kepala panggul \\ \hline
	Anemia defisiensi besi pada kehamilan & Intra-Uterine FetalDeath & IUFD & Persalinan preterm \\ \hline
	Ruptur uteri & Bayi post matur & Ketuban pecah dini & Distosia \\ \hline
	Malpresentasi & Partus lama & Prolaps tali pusat & Hipoksia janin \\ \hline
	Ruptur serviks & Ruptur perineum tingkat 1 & Ruptur perineum tingkat 2 & Ruptur perineum tingkat 3 \\ \hline
	Ruptur perineum tingkat 4 & Retensi plasenta & Inversio uterus & Perdarahan postpartum \\ \hline
	Tromboemboli & Endometritis & Inkontinensia urine & Inkontinensia feses \\ \hline
	Trombosis venadalam & Tromboflebitis & Subinvolusio uterus & Kista \\ \hline
	abses kelenjar bartolini & Abses folikel rambut & kelenjar sebasea & Malformasi kongenital \\ \hline
	Kistokel & Rektokel & Corpus alienum vaginae & Kista Gartner \\ \hline
	Fistula & Kista Nabotian & Polip serviks & Malformasi kongenital uterus \\ \hline
	Prolaps uterus, sistokel, rektokel & Hematokolpos & Endometriosis & Hiperplasia endometrium \\ \hline
	Menopause & perimenopausal syndrome & Polikistik ovarium & Kehamilan ektopik \\ \hline
	Karsinoma serviks & Karsinoma endometrium & Karsinoma ovarium & Teratoma ovarium \\ \hline
	kista dermoid & Kista ovarium & Torsi dan ruptur kista & Korio karsinoma Adenomyosis \\ \hline
	Malpresentasi & Inflamasi & abses & Mastitis \\ \hline
	Cracked nipple & Inverted nipple & Fibrokista & Fibroadenoma Mammae \\ \hline
	Tumor Filoides & Karsinoma payudara & Paget & Ginekomastia \\ \hline
	Infertilitas & Gangguan ereksi & Gangguan ejakulasi & Diabetes melitus tipe 1 \\ \hline
	Diabetes melitus tipe 2 & Ketoasidosis diabetikum nonketotik & Hiperglikemia hiperosmolar & Hipoglikemia ringan \\ \hline
	Hipoglikemia berat & Diabetes insipidus & Akromegali & gigantisme \\ \hline
	Defisiensi hormon pertumbuhan & Hiperparatiroid & Hipoparatiroid & Hipertiroid \\ \hline
	Tirotoksikosis & Hipotiroid & Goiter & Tiroiditis \\ \hline
	Cushing’s disease & Krisis adrenal & Addison’s disease & Pubertas prekoks \\ \hline
	Hipogonadisme & Prolaktinemia & Adenoma tiroid & Karsinoma tiroid \\ \hline
	Malnutrisi energi protein & Defisiensi vitamin & Defisiensi mineral & Dislipidemia \\ \hline
	Porfiria & Hiperurisemia & Obesitas & Sindrom metabolik \\ \hline
	Anemia & Anemia aplastik & Anemia defisiensi besi & Anemia hemolitik \\ \hline
	Anemia makrositik & Anemia megaloblastik & Hemoglobinopati & Polisitemia \\ \hline
	Gangguan pembekuan darah & trombositopenia & hemofilia & Von Willebrand’sdisease \\ \hline
	DIC & Agranulositosis & Inkompatibilitas golongan darah & Timoma \\ \hline
	Limfoma non Hodgkin’s & Hodgkin’s & Leukemia akut & Leukemia kronik \\ \hline
	Mieloma multipel & Limfadenopati & Limfadenitis & Bakteremia \\ \hline
	Demam dengue & DHF & Dengue shock syndrome & Malaria \\ \hline
	Leishmaniasis dan tripanosomiasis & Toksoplasmosis & Leptospirosis & Sepsis \\ \hline
	Lupus eritematosus sistemik & Poliarteritis nodosa & Polimialgia reumatik & Reaksi anafilaktik \\ \hline
	Demam reumatik & Artritis reumatoid & Juvenile chronic arthritis & Henoch-schoenlein purpura \\ \hline
	Eritema multiformis & Imunodefisiensi & Artritis & osteoarthritis \\ \hline
	Fraktur terbuka & Fraktur tertutup & Fraktur klavikula & Fraktur patologis \\ \hline
	Fraktur tulang belakang & dislokasi tulang belakang & Dislokasi padasendi ekstremitas & Osteogenesis imperfekta \\ \hline
	Ricketsia, osteomalasia & Osteoporosis & Akondroplasia & Displasia fibrosa \\ \hline
	Tenosinovitis supuratif & Tumor tulang primer & Tumor tulang sekunder & Osteosarkoma \\ \hline
	Sarcoma Ewing & Kista ganglion & Trauma sendi & skoliosis \\ \hline
	kifosis & lordosis & Spondilitis & spondilodisitis \\ \hline
	Teratoma sakrokoksigeal & Spondilolistesis & Spondilolisis & Lesi pada ligamentosa panggul \\ \hline
	Displasia panggul & Nekrosis kaputfemoris & Tendinitis Achilles & Ruptur tendon Achilles \\ \hline
	Lesi meniskus & Lesi medial & Lesi lateral & Instabilitas sendi tumit \\ \hline
	Malformasi kongenital & genovarum & genovalgum & club foot \\ \hline
	pes planus & Claw foot & drop foot & Claw hand \\ \hline
	drop hand & Ulkus pada tungkai & Osteomielitis & Rhabdomiosarkoma \\ \hline
	Leiomioma & leiomiosarkoma & liposarkoma & Lipoma \\ \hline
	Fibromatosis & fibroma & fibrosarkoma & Veruka vulgaris \\ \hline
	Kondiloma akuminatum & Moluskum kontagiosum & Herpes zoster & Morbili \\ \hline
	Varisela & Herpes simpleks & Impetigo & Impetigo ulseratif \\ \hline
	ektima & Folikulitis superfisialis & Furunkel & karbunkel \\ \hline
	Eritrasma & Erisipelas & Skrofuloderma & Lepra \\ \hline
	Reaksi lepra & Sifilis & Tinea kapitis & Tinea barbe \\ \hline
	Tinea fasialis & Tinea korporis & Tinea manus & Tinea unguium \\ \hline
	Tinea kruris & Tinea pedis & Pitiriasis vesikolor & Kandidosis mukokutan ringan \\ \hline
	Cutaneus larva migran & Filariasis & Pedikulosis kapitis & Pedikulosis pubis \\ \hline
	Skabies & Reaksi gigitan serangga & Dermatitis kontakiritan & Dermatitis kontakalergika \\ \hline
	Dermatitis atopik & Dermatitis numularis & Liken simpleks kronik & Liken simpleks neurodermatitis \\ \hline
	Napkin eczema & Psoriasis vulgaris & Dermatitis seboroik & Pitiriasis rosea \\ \hline
	Akne vulgaris ringan & Akne vulgaris sedang-berat & Hidradenitis supuratif & Dermatitis perioral \\ \hline
	Miliaria & Toxic epidermal necrolysis & Sindrom Stevens Johnson & Urtikaria akut \\ \hline
	Urtikaria kronis & Angioedema & Lupus eritematosis kulit & Ichthyosis vulgaris \\ \hline
	Exanthematous drug eruption & fixed drug eruption & Vitiligo & Melasma \\ \hline
	Albino & Hiperpigmentasi pasca inflamasi & Hipopigmentasi pasca inflamasi & Keratosis seboroik \\ \hline
	Kista epitel & Squamous cell carcinoma & Karsinoma sel skuamosa & Basal cell carcinoma \\ \hline
	Karsinoma sel basal & Xanthoma & Hemangioma & Lentigo \\ \hline
	Nevus pigmentosus & Melanoma maligna & Alopesia areata & Alopesia androgenik \\ \hline
	Telogen eflluvium & Psoriasis vulgaris & Vulnus laseratum & Vulnus punctum \\ \hline
	Vulnus perforatum & Vulnus penetratum &  &  \\ \hline
\end{longtable}