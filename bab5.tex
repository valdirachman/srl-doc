%!TEX root = skripsi.tex
%-----------------------------------------------------------------------------%
\chapter{\babLima}
%-----------------------------------------------------------------------------%

In this chapter, we describe the data statistics followed by experiment scenarios, results, and analyses.
%-----------------------------------------------------------------------------%

\section{Evaluation Metrics}

\section{Data Statistics}

\section{Experiment Design}

In this section, we present our experiment results and the analysis accordingly. There are two set of scenarios. The first scenario set aims to find the best combination set of features. This scenario consists of four combinations of three features, namely feature Word Embedding (WE), POS Tag (POS), and Word Embedding of Neighbors (WE-N) as presented in Table~\ref{tab:feature_scenario}.

The second scenario set evaluates two different architectures, which are the original BLSTM and CA-BLSTM. Underlying both architectures is a Convolutional Neural Network (CNN) layer, in order to catch more information surrounding each time step. This second scenario also aims to see the effect of hyper-parameter tuning for the CA-BLSTM.

As for evaluation, we use precision, recall and F1 metrics for all scenarios. The results are evaluated with partial match approach. To illustrate, suppose that the expected label consists of two words or more and the predicted contains only one word of it, partial match will still count it as a true positive.

\section{Scenario 1: Feature Selection}
Table~\ref{tab:feature_scenario} shows the scenario results of four feature combinations. The highest result is achieved with combination of WE + POS, followed by WE + WE-N + POS, WE, and WE + WE-N, with F1 scores of 72.29\%, 72.22\%, 62.00\%, and 61.92\% respectively. From these results, we can see the big impact POS Tag contributes for the performance. Using POS Tag enhances the result up to 10.29\%, when we compare WE + POS and WE combinations. The explanation would be the fact that POS Tag contains meaningful information such as Noun and Adjective which describes the word, it thus supports the word embedding feature when the training data size is relatively low.

Surprisingly, when we combine the neighboring words of the argument as in WE + WE-N, the result slightly decreases by 0.08\%, compared to only using WE feature. This is also the case when we compare WE + WE-N with WE + WE-N + POS scenarios, which decreases by 0.07\% when we used neighboring words. It shows us that neighboring words do not improve the performance at all. We suggest that this is because the CNN layer already extracts these information implicitly, by capturing surrounding information and compressed it into one vector. Hence, we do not need explicit neighboring words as part of our features. 

\begin{table}
	\caption{Results of Feature Combination Scenario}
	\label{tab:feature_scenario}
	\begin{tabular}{llll}
		Features		&Precision	&Recall		&F1			\\
		WE				&	64.68\%				&	60.25\%				&	62.00\%	\\
		WE + POS		&	74.24\%				&	\textbf{71.26\%}	&	\textbf{72.29\%}	\\
		WE + WE-N		&	64.17\%				&	60.29\%				&	61.92\%	\\
		WE + POS + WE-N	&	\textbf{74.51\%}	&	70.69\%				&	72.23\%	\\
	\end{tabular}
\end{table}

Since WE + POS outputs the best result in terms of F1 score in this scenario set, we will use it for the next set of scenarios.

\subsection{Scenario 2: Model Selection}
The experiment results of the second scenario set are presented in Table~\ref{tab:architecture_scenario}. First we show the result using the original BLSTM with F1 score of 72.29\%. On the next experiment, we added the attention mechanism on top of the BLSTM layer, with three different dimensions $K$ of weight matrix $W \in {\rm I\!R^{H \times K}}$ from the Equation~\ref{sum_weight}. The three dimensions of $K$ are 64, 128, and 256. When we experimented on the first dimension size of $K$, 64, it is notable that the F1 score increases by 0.76\%. The performance keeps increasing as we add the dimension size of $K$ until 256 with F1 score of 74.78\% which increased by 2.49\% compared to the original BLSTM. We also obtained the highest precision and recall with dimension size $K$ of 256. However, when we added more dimension above 256, the model seems to be overfitting.

The results show that the CA-BLSTM architecture outperforms the original BLSTM architecture. We suggest that our proposed model can capture context information at abstract level. We believe that the attention mechanism plays an essential role to extract which word has the most significant value as a hint to predict the semantic roles of each time step.

\begin{table}
	\caption{Results of Model Architecture Scenario}
	\label{tab:architecture_scenario}
	\begin{tabular}{llll}
		Name			&Precision					&Recall		&F1			\\
		BLSTM				&	74.24\%				&	71.26\%				&	72.29\%	\\
		CA-BLSTM-64			&	73.37\%				&	73.25\%				&	73.05\%	\\
		CA-BLSTM-128		&	76.25\%				&	73.52\%				&	74.05\%	\\
		CA-BLSTM-256		&	\textbf{77.03\%}	&	\textbf{73.55\%}	&	\textbf{74.78\%}\\
	\end{tabular}
\end{table}
\section{Metrik Evaluasi}
Pada eksperimen ini, untuk mendapatkan nilai akurasi dari masing-masing eksperimen \saya~menggunakan \textit{precision}, \textit{recall} dan \textit{f-measure}. \Saya~menggunakan \textit{10-cross fold validation} dalam menjalankan eksperimen. Terkait dengan penjelasan mengenai cara penghitungan dan evaluasi sudah dijelaskan pada Bab 3.

\section{Visualisasi Data}
Berikut merupakan visualisasi data dari korpus yang \saya~miliki. Dapat dilihat dari grafik \ref{fig:korpus}, bahwa persebaran jumlah entitas tidak seimbang. Hal ini menjadi kendala \saya~dalam melakukan penelitian ini, karena keterbatasan \textit{resource} dan tenaga untuk melakukan pelabelan dokumen secara manual.
\begin{figure}
	\centering
	\includegraphics[width=0.85\linewidth]{images/histogramentitaskorpus}
	\caption{Histogram Jumlah Entitas pada Korpus}
	\label{fig:korpus}
\end{figure}

Tabel \ref{diseasecorpus} menunjukkan 39 daftar entitas \textit{disease} teratas yang terdapat di dalam korpus.
\begin{table}
	\caption{Tabel Beberapa Entitas \textit{Disease} pada Korpus dan Jumlahnya}
	\label{diseasecorpus}
	\resizebox{\textwidth}{!}{%
		\begin{tabular}{|l|l|l|}
			\hline
			asma (20)                                  & isk (20)                         & tbc (18)                             \\ \hline
			infeksi (17)                               & sinusitis (16)                   & jerawat (15)                         \\ \hline
			alergi (14)                                & oligomenorea (14)                & maag (12)                            \\ \hline
			fam (11)                                   & flu (11)                         & kanker serviks (11)                  \\ \hline
			hiv (11)                                   & penyakit jantung (11)            & varikokel (10)                       \\ \hline
			wasir (10)                                 & infeksi saluran kemih (9)        & obesitas (9)                         \\ \hline
			cacar air (9)                              & diabetes (9)                     & albino (9)                           \\ \hline
			gerd (8)                                   & hepatitis b (8)                  & tth (8)                              \\ \hline
			liver (8)                                  & sifilis (8)                      & tuberkulosis (7)                     \\ \hline
			stroke (7)                                 & sakit maag (7)                   & anemia aplastik (7)                  \\ \hline
			alergi susu sapi (7)                       & hepatitis (7)                    & diare (6)                            \\ \hline
			scabies (6)                                & kanker otak (6)                  & jengger ayam (6)                     \\ \hline
			pid (6)                                    & osteoporosis (5)                 & tifus (5)                            \\ \hline
		\end{tabular}
	}
\end{table}

Tabel \ref{symtomcorpus} menunjukkan 39 daftar entitas \textit{symptom} teratas yang terdapat di dalam korpus.

% Please add the following required packages to your document preamble:
% \usepackage[normalem]{ulem}
% \useunder{\uline}{\ul}{}
\begin{table}
	\caption{Tabel Beberapa Entitas \textit{Symptom} pada Korpus dan Jumlahnya}
	\label{symtomcorpus}
	\resizebox{\textwidth}{!}{%
		\begin{tabular}{|l|l|l|}
			\hline
			nyeri (60)                                                     & sakit kepala (51)                                       & demam (50)                                                \\ \hline
			mual (40)                                                      & gatal (30)                                              & batuk (27)                                                \\ \hline
			pusing (24)                                                    & muntah (23)                                             & diare (16)                                                \\ \hline
			keputihan (15)                                                 & nyeri dada (14)                                         & sesak nafas (14)                                          \\ \hline
			kejang (13)                                                    & keringat dingin (13)                                    & pilek (12)                                                \\ \hline
			nyeri kepala (10)                                              & stres (10)                                              & stress (10)                                               \\ \hline
			sesak (8)                                                      & jantung berdebar - debar (7)                            & flu (7)                                                   \\ \hline
			rasa nyeri (7)                                                 & lemas (7)                                               & kesemutan (7)                                             \\ \hline
			bersin (6)                                                     & sakit gigi (6)                                          & mimisan (6)                                               \\ \hline
			nyeri pinggang (6)                                             & depresi (5)                                             & perih (5)                                                 \\ \hline
			sakit perut (5)                                                & anyang - anyangan (4)                                   & bintilan (4)                                              \\ \hline
			kram perut (4)                                                 & pingsan (4)                                             & susah tidur (4)                                           \\ \hline
			rasa gatal (4)                                                 & perubahan mood (4)                                      & kelelahan (4)                                             \\ \hline
			
		\end{tabular}
	}
\end{table}


Tabel \ref{treatmentCorpus} menunjukkan 39 daftar entitas \textit{treatment} teratas yang terdapat di dalam korpus.

% Please add the following required packages to your document preamble:
% \usepackage{graphicx}
\begin{table}
	\caption{Tabel Beberapa Entitas \textit{Treatment} pada Korpus dan Jumlahnya}
	\label{treatmentCorpus}
	\resizebox{\textwidth}{!}{%
		\begin{tabular}{|l|l|l|}
			\hline
			operasi (40) & pemeriksaan fisik (34) & terapi (21) \\ \hline
			usg (15) & pemeriksaan penunjang (11) & tes darah (10) \\ \hline
			istirahat (9) & pemeriksaan darah (8) & pemeriksaan jantung (7) \\ \hline
			ct scan (7) & ekg (6) & x - ray (6) \\ \hline
			foto rontgen (5) & biopsi (5) & imunisasi (5) \\ \hline
			tes pap (5) & operasi sesar (4) & mri (4) \\ \hline
			istirahat cukup (4) & kontrasepsi (4) & dinebu (3) \\ \hline
			tes lbc (3) & minum air putih (3) & rontgen (3) \\ \hline
			rontgen dada (3) & susu (3) & fisioterapi (3) \\ \hline
			tes iva (3) & berolahraga (3) & tes penunjang (3) \\ \hline
			hindari memencet (2) & test pack (2) & terapi obat - obatan (2) \\ \hline
			makan makanan bergizi (2) & pasang iud (2) & cek darah (2) \\ \hline
			kompres dingin (2) & tes hpv (2) & gunakan kondom (2) \\ \hline
		\end{tabular}%
	}
\end{table}

Tabel \ref{drugCorpus} menunjukkan 39 daftar entitas \textit{drug} teratas yang terdapat di dalam korpus.
% Please add the following required packages to your document preamble:
% \usepackage{graphicx}
\begin{table}
	\caption{Tabel Beberapa Entitas \textit{Drug} pada Korpus dan Jumlahnya}
	\label{drugCorpus}
	\resizebox{\textwidth}{!}{%
		\begin{tabular}{|l|l|l|}
			\hline
			antibiotik (31) & paracetamol (8) & ibuprofen (7) \\ \hline
			parasetamol (5) & whey protein (5) & obat tetes mata (5) \\ \hline
			obat batuk (5) & vitamin c (4) & obat herbal (4) \\ \hline
			nebulizer (4) & kiranti (4) & salbutamol (4) \\ \hline
			salep (4) & obat antibiotik (4) & amoxilin (3) \\ \hline
			metronidazol (3) & obat pelangsing (3) & nitrokaf (3) \\ \hline
			kortikosteroid (3) & asam mefenamat (3) & tetrahydrolipstatin (3) \\ \hline
			steroid (3) & ctm (3) & asam valproat (3) \\ \hline
			rifampicin (3) & obat pereda nyeri (2) & habbatussauda (2) \\ \hline
			obat depaken (2) & glucovance (2) & obat pereda rasa nyeri (2) \\ \hline
			obat nyeri (2) & obat penurun panas (2) & probiotik (2) \\ \hline
			sunblock (2) & obat jerawat (2) & bicrolid (2) \\ \hline
			plembab (2) & aspirin (2) & acyclovir (2) \\ \hline
		\end{tabular}%
	}
\end{table}

%-----------------------------------------------------------------------------%
\section{Desain Eksperimen}
Pada penelitian ini, \saya~melakukan 3 buah skenario utama, yaitu skenario untuk melakukan implementasi dan eksperimen ulang pada \textit{baseline} (penelitian sebelumnya), skenario untuk menguji fitur yang memiliki kontribusi untuk meningkatkan akurasi dari setiap eksperimen dan skenario untuk menguji arsitektur RNNs yang \saya~usulkan. Berikut merupakan skenario yang \saya~rancang dalam penelitian ini:
\begin{enumerate}
	\item Skenario 1: Skenario untuk mengimplementasikan ulang eksperimen sebelumnya\\
	Skenario ini bertujuan untuk mengimplementasikan ulang model yang diusulkan pada penelitian \cite{skripsiKakRadit}, yaitu dengan menggunakan model \textit{Conditional Random Fields} (CRFs). Fitur yang digunakan merupakan fitur yang memberikan hasil terbaik pada penelitian sebelumnya, yaitu fitur kata, fitur kamus, fitur frasa, fitur panjang kata dan fitur kata sebelum. Tujuan dari skenario ini yaitu sebagai \textit{baseline} dan pembanding pada penelitian ini.
	
	\item Skenario 2: Skenario untuk menguji fitur\\
	Skenario ini bertujuan untuk mendapatkan kombinasi fitur terbaik sehingga memberikan akurasi terbaik. \Saya~mencoba masing-masing fitur dengan menggunakan model arsitektur LSTMs 1 layer. Apabila penggunaan fitur memberikan hasil yang lebih dari pada hasil eksperimen sebelumnya, fitur tersebut akan dipertahankan untuk eksperimen yang selanjutnya. Skenario ini memiliki 9 sub-skenario, yaitu:
	\begin{enumerate}
		\item Sub-skenario 2.1 Fitur kata
		\item Sub-skenario 2.2 Fitur kata dan kamus
		\item Sub-skenario 2.3 Fitur kata, kamus dan \textit{stopword}
		\item Sub-skenario 2.4 Fitur kata, kamus, \textit{stopword} dan POS-Tag
		\item Sub-skenario 2.5 Fitur kata, kamus, \textit{stopword}, POS-Tag dan frasa kata
		\item Sub-skenario 2.5 Fitur kata, kamus, \textit{stopword} dan frasa kata
		\item Sub-skenario 2.5 Fitur kata, kamus, \textit{stopword}, frasa kata dan kata sebelum
		\item Sub-skenario 2.5 Fitur kata, kamus, \textit{stopword}, frasa kata, kata sebelum dan kata sesudah
	\end{enumerate}

	\item Skenario 3: Skenario untuk menguji arsitektur RNNs\\
	Skenario ini bertujuan untuk melihat pengaruh arsitektur RNNs pada penelitian ini. \Saya~mencoba kedua arsitektur RNNs yang telah diusulkan sebelumnya dengan menggunakan kombinasi fitur terbaik dari eksperimen di skenario pengujian fitur di atas. Pada skenario ini, terdapat 2 sub-skenario, yaitu:
	\begin{enumerate}
		\item Sub-skenario 3.1: LSTMs 1 layer
		\item Sub-skenario 3.2: LSTMs 2 layer multi-\textit{input}
	\end{enumerate}
	
\end{enumerate}

\section{Skenario 1: \textit{Baseline} Eksperimen \cite{skripsiKakRadit}}
Pada penelitian ini, \saya~mencoba melakukan implementasi ulang penelitian yang dilakukan oleh \cite{skripsiKakRadit}. Data yang digunakan adalah data yang \saya~gunakan dalam penelitian ini supaya perbandingan yang didapatkan \textit{apple-to-apple}. Implementasi dan eksperimen ini bertujuan sebagai \textit{baselaine} eksperimen dan penelitian yang \saya~lakukan. Selain itu, juga untuk mengetahui secara singkat fitur yang diskriminatif dalam melakukan \textit{sequence labeling} pada \mer~ini. Model yang digunakan sama dengan penelitian \cite{skripsiKakRadit}, yaitu CRFs dengan penggunaan fitur kata, fitur kamus, fitur frasa, fitur panjang kata dan fitur kata sebelum.

\subsection{Hasil Eksperimen}
\textbf{Waktu komputasi}: $ 6093.0 $ detik.

Berikut merupakan hasil implementasi ulang penelitian yang dilakukan oleh \cite{skripsiKakRadit}.
	
\begin{table}
	\centering
	\caption{Tabel Hasil Eksperimen dari Penelitian \cite{skripsiKakRadit} (\textit{Baseline})}
	\begin{tabular}{|c|c|c|c|c|}
		\hline
							& \textit{Precission} & \f{\f{Recall}} & \f{\f{F-Measure}} \\ \hline
		\textit{Disease}    & 63.68\%             & 55.45\%        & 59.13\%           \\ \hline
		\textit{Symptom}    & 61.43\%             & 59.21\%        & 60.18\%           \\ \hline
		\textit{Treatment}  & 53.10\%             & 45.97\%        & 48.82\%           \\ \hline
		\textit{Drug}		& 58.99\%             & 44.46\%        & 48.23\%           \\ \hline
		\textit{\textbf{Overall}}&\textbf{59.03\%}  & \textbf{51.27\%}& \textbf{54.09\%} \\ \hline
	    \end{tabular}
\label{table:radit}
\end{table}

\begin{figure}
	\centering
	\includegraphics[width=0.85\linewidth]{images/radit}
	\caption{Histogram Metrik Evaluasi dari Penelitian \cite{skripsiKakRadit} (\textit{Baseline})}
	\label{fig:radit}
\end{figure}


\section{Skenario 2: Skenario Pengujian Fitur}
		  
	\subsection{Eksperimen 2.1: Fitur Kata}
	%-----------------------------------------------------------------------------%
	Merujuk pada penelitian \cite{mujiono2016new}, penelitian tersebut bertujuan untuk mendapatkan \textit{non-handcrafted feature}, yaitu fitur kata itu sendiri dengan menggunakan \textit{tools Word Embedding}. Oleh karena itu, \saya~menguji fitur ini untuk mengetahui pengaruhnya pada program \mer~di penelitian ini. 
	
	\subsubsection{Hasil Eksperimen}
	\textbf{Waktu komputasi}: $ 5191.0 $ detik.
	
	Tabel \ref{table:ekskata} menampilkan hasil pelabelan otomatis dengan menggunakan fitur kata itu sendiri yang direpresentasikan dengan menggunakan vektor \textit{word embedding}.
	
	% Please add the following required packages to your document preamble:
	% \usepackage{multirow}
	\begin{table}
		\centering
		\caption{Tabel Hasil Eksperimen 2.1 dibandigkan dengan \textit{Baseline}}
		\begin{tabular}{|l|c|c|c|c|c|c|}
			\hline
			\multicolumn{1}{|c|}{\multirow{2}{*}{Entitas}} & \multicolumn{3}{c|}{\textit{Baseline (Herwando 2016)}} & \multicolumn{3}{c|}{Eksperimen 2.1} \\ \cline{2-7} 
			\multicolumn{1}{|c|}{} & \textit{Precision} & \textit{Recall} & \textit{F-Measure} & \textit{Precision} & \textit{Recall} & \textit{F-Measure} \\ \hline
			\textit{Disease} & 63.68\% & 55.45\% & 59.13\% & 61.38\% & 60.42\% & 60.37\% \\ \hline
			\textit{Symptom} & 61.43\% & 59.21\% & 60.18\% & 57.05\% & 56.13\% & 56.19\% \\ \hline
			\textit{Treatment} & 53.10\% & 45.97\% & 48.82\% & 49.92\% & 47.17\% & 46.96\% \\ \hline
			\textit{Drug} & 58.99\% & 44.46\% & 48.23\% & 62.86\% & 53.32\% & 57.28\% \\ \hline
			\textbf{Overall} & \textbf{59.30\%} & \textbf{51.27\%} & \textbf{54.09\%} & \textbf{57.80\%} & \textbf{54.26\%} & \textbf{55.20\%} \\ \hline
		\end{tabular}
	\label{table:ekskata}
	\end{table}

	\begin{figure}
	  \centering
	  \includegraphics[width=0.85\linewidth]{images/histogram00}
	  \caption{Histogram Perbandingan \textit{F-measure} \textit{Baseline} dengan Eksperimen 2.1}
	  \label{fig:ekskata}
	\end{figure}
		
	\subsubsection{Analisis}
	Pada tabel \ref{table:ekskata}, dapat dilihat bahwa secara umum \textit{recall} dan \textit{F-measure} yang dihasilkan lebih baik dibandingkan dengan \textit{baseline}, walaupun untuk beberapa entitas nilainya lebih rendah (entitas \textit{symptom} dan \textit{treatment}). Selain itu, apabila dilihat pada grafik \ref{fig:ekskata}, secara umum model ini memberikan hasil yang lebih baik pada entitas \textit{disease} dan \textit{drug}. Kemudian rata-rata \textit{F-measure} yang didapatkan yaitu $ 55.20\% $, lebih tinggi dibandingkan \textit{baseline} yang \saya~gunakan yaitu $ 54.09\% $. Hal ini sangat menarik karena hanya dengan penggunaan fitur kata saja, hasil yang diberikan secara umum lebih baik dibandingkan \textit{baseline}.
	
	Pada eksperimen ini, ada beberapa entitas masih memiliki nilai \textit{precision, recall} dan \textit{f-measure} lebih kecil apabila dibandingkan dengan hasil yang dicapai \cite{skripsiKakRadit}. Setelah \saya~melakukan analisis terhadap penggunaan penggunaan \textit{tools Word Embedding}, ternyata terdapat 429 kata unik yang tidak terdapat dalam model \textit{word embedding}. Hal ini disebabkan oleh beberapa hal, yaitu:
	\begin{enumerate}
		\item Terdapat kata di dalam korpus yang tidak baku atau salah eja\\
		Korpus yang didapatkan berasal dari forum kesehatan \textit{online} yang bersifat non-formal. Oleh karena itu baik pasien maupun dokter bebas mengutarakan pendapatnya tanpa adanya aturan bahasa formal. Oleh karena itu banyak ditemukan adanya kata yang tidak baku atau salah eja, misalnya:
		\begin{itemize}
			\item dllsebaiknya,
			\item sekatrang,
			\item infeksiny,
			\item kliengan.
		\end{itemize}
		
		\item Terdapat istilah sulit yang tidak terkandung di dalam model\\
		Terdapat beberapa istilah kesehatan pada korpus yang tidak ada di dalam model, hal ini disebabkan karena data untuk \textit{training} model \textit{word embedding} terbatas (model sangat tergantung pada data \textit{training}). Oleh karena terdapat beberapa kata yang tidak terdaftar di dalam model. Contoh beberapa istilah sulit tersebut yaitu:
		\begin{itemize}
			\item microdermabrians,
			\item flixotide,
			\item bimaflox,
			\item polysiloxanes,
			\item scizophrenia.
		\end{itemize}
	
		\item Terdapat kata yang merupakan nama orang\\
		Adanya kata yang merupakan nama orang tidak bisa dihindari di dalam forum kesehatan \textit{online}. Selain itu, nama orang berbeda untuk setiap orang sehingga sulit mendapatkan vektor kata nama orang tersebut di dalam \textit{word embedding}. Contoh dari kata yang merupakan nama orang di dalam korpus yaitu:
		\begin{itemize}
			\item novira,
			\item risma,
			\item oktavia,
			\item sudianto.
		\end{itemize}
	\end{enumerate}
	
	Dari beberapa kasus di atas, \saya~mengusulkan menambahkan fitur yang memperkaya informasi dari fitur kata itu sendiri, misalnya seperti apakah suatu kata terdapat dalam sebuah kamus kesehatan, informasi POS-Tag atau informasi yang lain. Oleh karena itu, \saya~mencoba menggunakan tambahan fitur lain untuk meningkatkan akurasi pada penelitian ini, yaitu pada sub-eksperimen \ref{eks:subeksdict}.
	
	%-----------------------------------------------------------------------------%
	\subsection{Eksperimen 2.2: Fitur Kata dan Kamus Kesehatan (\textit{Disease, Symptom, Treatment} dan \textit{Drug})}\label{eks:subeksdict}
	%-----------------------------------------------------------------------------%
	Pada sub-eksperimen ini, \saya~menggunakan tambahan fitur Kamus Kesehatan karena berdasarkan penelitian \cite{skripsiKakRadit} fitur ini memiliki konribusi untuk menambah akurasi pada sistem \mer. Selain itu, menurut \saya, informasi suatu kata terdapat dalam sebuah kamus kesehatan mungkin akan memberikan kontribusi untuk meningkatkan akurasi. Oleh karena itu, \saya~mencoba untuk menambahkan fitur ini ke dalam model RNNs.
	
	\subsubsection{Hasil Eksperimen}
	\textbf{Waktu komputasi}: $ 5658.0 $ detik.
	
	Tabel \ref{table:ekskamus} merupakan tabel hasil eksperimen yang didapatkan dengan menggunakan fitur ini.
	
	% Please add the following required packages to your document preamble:
	% \usepackage{multirow}
	\begin{table}
		\centering
		\caption{Tabel Hasil Eksperimen 2.2 dibandigkan dengan \textit{Baseline}}
		\label{table:ekskamus}
		\begin{tabular}{|l|c|c|c|c|c|c|}
			\hline
			\multicolumn{1}{|c|}{\multirow{2}{*}{Entitas}} & \multicolumn{3}{c|}{Baseline (Herwando 2016)} & \multicolumn{3}{c|}{Eksperimen 2.2} \\ \cline{2-7} 
			\multicolumn{1}{|c|}{} & \textit{Precision} & \textit{Recal} & \textit{F-Measure} & \textit{Precision} & \textit{Recal} & \textit{F-Measure} \\ \hline
			\textit{Disease} & 63.68\% & 55.45\% & 59.13\% & 67.32\% & 61.78\% & 64.10\% \\ \hline
			\textit{Symptom} & 61.43\% & 59.21\% & 60.18\% & 60.55\% & 55.12\% & 57.41\% \\ \hline
			\textit{Treatment} & 53.10\% & 45.97\% & 48.82\% & 52.21\% & 44.18\% & 47.02\% \\ \hline
			\textit{Drug} & 58.99\% & 44.46\% & 48.23\% & 59.42\% & 59.71\% & 57.90\% \\ \hline
			\textit{\textbf{Overall}} & \textbf{59.30\%} & \textbf{51.27\%} & \textbf{54.09\%} & \textbf{59.88\%} & \textbf{55.20\%} & \textbf{56.61\%} \\ \hline
		\end{tabular}
	\end{table}
	
	Berikut merupakan grafik yang menunjukkan perbandingan \textit{F-Measure} eksperimen \ref{table:ekskamus} dengan \textit{baseline} dalam bentuk histogram.
	
	\begin{figure}
		\centering
		\includegraphics[width=0.85\linewidth]{images/histogram2}
		\caption{Histogram Perbandingan \textit{F-measure} \textit{Baseline} dengan Eksperimen 2.2}
		\label{fig:ekskamus}
	\end{figure}

	\subsubsection{Analisis}

	Dari tabel dan grafik \ref{fig:ekskamus}, didapatkan informasi bahwa dengan menggunakan tambahan fitur kamus kesehatan terlihat bahwa entitas \textit{Disease} mengalami kenaikan nilai \textit{precision}, \textit{recall}, dan \textit{f-measure}. Selain itu, entitas \textit{symptom} dan \textit{tratment} mengalami kenaikan nilai \textit{precision} dan \textit{f-measure}. Entitas \textit{drug} mengalami penurunan pada nilai \textit{precision} namun mengalami kenaikan pada nilai \textit{recall} dan \textit{f-measure}-nya. Secara keseluruhan,  Sedangkan entitas \textit{drug} memiliki \textit{precission} tertinggi, yaitu 62.86\%. Grafik \ref{fig:ekskamus} menunjukkan perbandingan \textit{precision}, \textit{recall} dan \textit{f-measure} untuk masing-masing entitas.
	
	Dari analisis yang \saya~lakukan terhadap korpus dan kamus kesehatan, terdapat beberapa entitas pada korpus yang tidak terdapat pada kamus sehingga mengakibatkan kenaikan hasil tidak besar. Hal ini karena terdapat beberapa penyebab, yaitu:
	\begin{enumerate}
		\item Ada entitas yang merupakan kombinasi atau gabungan dari entitas lain yang dihubungkan dengan kata penghubung. Hal ini banyak \saya~temukan pada entitas \textit{treatment} dan \textit{symptom}. Contoh dari kasus ini yaitu:
		\begin{itemize}
			\item nyeri hebat dibagian ulu hati dan pinggang belakang (gabungan dari "nyeri hebat dibagian ulu hati" dan "nyeri hebat pinggang belakang")
			\item kondisi fases berampas , kuning , sedikit berlendir (gabungan dari "kondisi fases berampas", "kondisi fases kuning" dan "kondisi fases sedikit berlendir")
			\item alis atas dan bibir tidak bisa digerakkan (gabungan dari "alis atas tidak bisa digerakkan" dan "bibir tidak bisa digerakkan")
		\end{itemize}
		
		\item Penggunaan kata ganti orang di dalam entitas. Contoh dari kasus ini yaitu:
		\begin{itemize}
			\item suara \textbf{saya} hilang
			\item gusi \textbf{saya} berdarah
			\item pinggang \textbf{saya} sakit
		\end{itemize}
		
		\item Kesalahan eja pada entitas atau penggunaan kata yang tidak baku, misalnya:
		\begin{itemize}
			\item radang paru 2
			\item butawarna
			\item jrawatan
			\item kanker darah setadium 1
		\end{itemize}
	\end{enumerate}

	Dibandingkan dengan hasil eksperimen \cite{skripsiKakRadit}, hasil yang dicapai pada eksperimen ini masih lebih rendah pada entitas \textit{symptom} dan \textit{treatment}. Menurut \saya~perlu ada informasi tambahan untuk meningkatkan akurasi. Seperti yang kita ketahui bahwa eksperimen \cite{skripsiKakRadit} tidak hanya menggunakan fitur kata itu sendiri dan kamus kesehatan saja. Oleh karena itu, \saya~mencoba melakukan eksperimen kembali dengan menggunakan tambahan fitur lain pada sub-eksperimen \ref{eks:subekstopword}.
	
	%-----------------------------------------------------------------------------%
	\subsection{Eksperimen 2.3: Fitur Kata, Kamus Kesehatan dan \textit{Stopword}}\label{eks:subekstopword}
	%-----------------------------------------------------------------------------%
	Pada sub-eksperimen ini. \saya~mencoba menambahkan informasi lain berupa fitur yang berisi sebuah kata apakah terdapat di dalam kamus \textit{stop word} atau tidak. \Saya~berpendapat bahwa dengan adanya informasi \textit{stop word}, adanya kesalahan suatu kata tidak berentitas yang dilabeli sebagai kata berentitas oleh model dapat dikurangi.
	
	\subsubsection{Hasil Eksperimen}
	\textbf{Waktu komputasi}: $ 6019.5 $ detik.
	
	Rangkuman hasil sub-eksperimen ini dapat dilihat di Tabel \ref{table:eksstop} dan Gambar \ref{fig:eksstop}.
	

	\begin{table}
		\centering
		\caption{Tabel Hasil Eksperimen 2.3 dibandigkan dengan \textit{Baseline}}
		\begin{tabular}{|l|c|c|c|c|c|c|}
			\hline
			\multicolumn{1}{|c|}{\multirow{2}{*}{Entitas}} & \multicolumn{3}{c|}{Baseline (Herwando 2016)} & \multicolumn{3}{c|}{Eksperimen 2.3} \\ \cline{2-7} 
			\multicolumn{1}{|c|}{} & \textit{Precision} & \textit{Recall} & \textit{F-Measure} & \textit{Precision} & \textit{Recall} & \textit{F-Measure} \\ \hline
			\textit{Disease} & 63.68\% & 55.45\% & 59.13\% & 65.97\% & 59.81\% & 62.28\% \\ \hline
			\textit{Symptom} & 61.43\% & 59.21\% & 60.18\% & 63.08\% & 55.20\% & 58.68\% \\ \hline
			\textit{Treatment} & 53.10\% & 45.97\% & 48.82\% & 54.73\% & 46.27\% & 49.69\% \\ \hline
			\textit{Drug} & 58.99\% & 44.46\% & 48.23\% & 61.88\% & 58.99\% & 59.57\% \\ \hline
			\textit{\textbf{Overall}} & \textbf{59.30\%} & \textbf{51.27\%} & \textbf{54.09\%} & \textbf{61.42\%} & \textbf{55.07\%} & \textbf{57.56\%} \\ \hline
		\end{tabular}
		\label{table:eksstop}
	\end{table}
		
	\begin{figure}
		\centering
		\includegraphics[width=0.85\linewidth]{images/histogram3}
		\caption{Histogram Perbandingan \textit{F-measure} \textit{Baseline} dengan Eksperimen 2.3}
		\label{fig:eksstop}
	\end{figure}

	\subsubsection{Analisis}
	
	Dari Tabel \ref{table:eksstop} dan Gambar \ref{fig:eksstop} dapat diamati bahwa secara umum, penggunaan fitur kamus \textit{stop word} dapat meningkatkan \textit{precision, recall,} dan \textit{f-measure}. Untuk lebih detailnya, entitas \textit{disease} mengalami penurunan nilai \textit{precision} dan \textit{f-measure} tetapi mengalami kenaikan nilai \textit{recall}. Entitas \textit{symptom} dan \textit{treatment} mengalami kenaikan untuk nilai \textit{precision, recall} dan \textit{f-measure}. Sedangkan entitas \textit{drug} mengalami kenaikan pada nilai \textit{precission} dan \textit{f-measure} teteapi mengalami penurunan pada nilai \textit{recall}.
		
	Pada sub-eksperimen ini, walaupun secara umum akurasi lebih baik dibandingkan dengan sub-eksperimen sebelumnya, hasil sub-eksperimen ini masih lebih rendah pada entitas \textit{treatment} apabila dibandingkan dengan hasil eksperimen \cite{skripsiKakRadit}. Oleh karena \saya~mengusulkan fitur tambahan lain yaitu fitur POS-Tag yang akan dijelaskan pada sub-eksperimen \ref{eks:subekspostag}.
	
	%-----------------------------------------------------------------------------%
	\subsection{Eksperimen 2.4: Fitur Kata, Kamus Kesehatan, \textit{Stopword} dan POS-Tag}\label{eks:subekspostag}
	%-----------------------------------------------------------------------------%
	Pada sub-eksperimen ini, \saya~menambahkan informasi baru pada \textit{resource} yang akan digunakan untuk \textit{training} model yang berupa fitur POS-Tag. Sebelumnya fitur ini telah digunakan pada penelitian \cite{abacha2011medical} pada dokumen berbahasa Inggris dan memberikan kontribusi meningkatkan akurasi dari model \mer~yang dibangun. Oleh karena itu pada eksperimen ini \saya~mencoba menggunakan fitur tersebut dan ingin mengetahui apakah fitur POS-Tag memiliki kontribusi untuk meningkatkan akurasi pada \mer~dengan dokumen berbahasa Indonesia. 
	
	\subsubsection{Hasil Eksperimen}
	\textbf{Waktu komputasi}: $ 6952.0 $ detik.
	
	Rangkuman hasil sub-eksperimen ini dapat dilihat pada Tabel \ref{table:owndict4} dan Gambar \ref{fig:owndict4}.
	
	\begin{table}
		\centering
		\caption{Tabel Hasil Eksperimen 2.4 dibandigkan dengan \textit{Baseline}}
		\begin{tabular}{|l|c|c|c|c|c|c|}
			\hline
			\multicolumn{1}{|c|}{\multirow{2}{*}{Entitas}} & \multicolumn{3}{c|}{Baseline (Herwando 2016)} & \multicolumn{3}{c|}{Eksperimen 2.4} \\ \cline{2-7} 
			\multicolumn{1}{|c|}{} & \textit{Precision} & \textit{Recall} & \textit{F-Measure} & \textit{Precision} & \textit{Recall} & \textit{F-Measure} \\ \hline
			\textit{Disease} & 63.68\% & 55.45\% & 59.13\% & 69.10\% & 58.67\% & 63.22\% \\ \hline
			\textit{Symptom} & 61.43\% & 59.21\% & 60.18\% & 61.09\% & 54.43\% & 57.00\% \\ \hline
			\textit{Treatment} & 53.10\% & 45.97\% & 48.82\% & 59.73\% & 44.10\% & 49.87\% \\ \hline
			\textit{Drug} & 58.99\% & 44.46\% & 48.23\% & 62.00\% & 55.74\% & 57.87\% \\ \hline
			\textit{\textbf{Overall}} & \textbf{59.30\%} & \textbf{51.27\%} & \textbf{54.09\%} & \textbf{62.98\%} & \textbf{53.24\%} & \textbf{56.99\%} \\ \hline
		\end{tabular}
		\label{table:owndict4}
	\end{table}
	
	\begin{figure}
		\centering
		\includegraphics[width=0.85\linewidth]{images/histogram4}
		\caption{Histogram Perbandingan \textit{F-measure} \textit{Baseline} dengan Eksperimen 2.4}
		\label{fig:owndict4}
	\end{figure}
	
	\subsubsection{Analisis}
	Dari tabel dan grafik di atas, entitas \textit{disease} dan \textit{treatment} memiliki nilai \textit{precision} dan \textit{f-measure} yang meningkat, tetapi dengan nilai \textit{recall} yang turun. Untuk entitas \textit{symptom}, nilai \textit{precision, recall}, dan \textit{f-measure} mengalami penurunan. Sedangkan entitas \textit{drug} mengalami kenaikan hanya pada \textit{precision}-nya saja.
	
	Hasil dari penggunaan fitur ini kurang baik, karena beberapa hal, yaitu:
	\begin{enumerate}
		\item Tag yang dihasilkan tidak konsisten. Pada beberapa entitas, terkadang suatu kata mendapatkan tag "A", namun di entitas yang lain untuk kata yang sama mendapatkan tag yang berbeda. Contoh dari kasus ini yaitu:
		\begin{itemize}
			\item "antibiotik" memiliki tag "vb", sedangkan pada entitas lain "antibiotik" memiliki tag "NN"
			\item "sakit kepala" memiliki beberapa tag pada entitas berbeda, yaitu "CD NN", "JJ NN", dan "NN NN"
			\item "nyeri" memiliki beberapa tag pada entitas berbeda, yaitu "NN", "VB", "IN", "WH", dan "IN".
		\end{itemize}
		\item Tidak ada perbedaan tag antara kata berentitas dengan tidak, misalnya nama orang mendapatkan tag "NN" (intan\_NN lusia\_NN), namun nama penyakit juga mendapatkan tag "NN" (Kanker\_NN Otak\_NN).
		\item Model POS-Tagger yang digunakan merupakan model untuk kalimat dengan topik umum, tidak dikhususkan pada topik kesehatan.
	\end{enumerate}
	
	Oleh karena itu pada sub-eksperimen selanjutnya, \saya~mencoba menambahkan fitur lain yang lebih spesifik dibandingkan dengan fitur POS-Tag, yaitu fitur Frasa Kata. Penjelasan lebih lanjut akan dibahas pada sub-eksperimen \ref{eks:subeksfrasa}.
	
	%-----------------------------------------------------------------------------%
	\subsection{Eksperimen 2.5: Fitur Kata, Kamus Kesehatan, \textit{Stopword}, POS-Tag dan Frasa Kata}\label{eks:subeksfrasa}
	%-----------------------------------------------------------------------------%
	Pada sub-eksperimen ini \saya~menambahkan fitur baru yaitu fitur Frasa Kata. Seperti yang telah dijelaskan pada Bab 3, entitas \textit{symptom} dan \textit{treatment} diharapkan akan lebih mudah dikenali karena pada umumnya merupakan frasa kata kerja. Sedangkan entitas \textit{disease} dan \textit{drug} diharapkan juga akan lebih mudah dikenali karena pada umumnya merupakan frasa kata benda.
	
	\subsubsection{Hasil Eksperimen}
	\textbf{Waktu komputasi}: $ 7528.5 $ detik.
	
	Rangkuman hasil sub-eksperimen ini dapat dilihat pada Tabel \ref{table:owndict4} dan Gambar \ref{fig:owndict4}.
	
	\begin{table}
		\centering
		\caption{Tabel Hasil Eksperimen 2.5 dibandigkan dengan \textit{Baseline}}
		\begin{tabular}{|l|c|c|c|c|c|c|}
			\hline
			\multicolumn{1}{|c|}{\multirow{2}{*}{Entitas}} & \multicolumn{3}{c|}{Baseline (Herwando 2016)} & \multicolumn{3}{c|}{Eksperimen 2.5} \\ \cline{2-7} 
			\multicolumn{1}{|c|}{} & \textit{Precision} & \textit{Recall} & \textit{F-Measure} & \textit{Precision} & \textit{Recall} & \textit{F-Measure} \\ \hline
			\textit{Disease} & 63.68\% & 55.45\% & 59.13\% & 67.49\% & 61.56\% & 63.81\% \\ \hline
			\textit{Symptom} & 61.43\% & 59.21\% & 60.18\% & 62.89\% & 52.27\% & 56.72\% \\ \hline
			\textit{Treatment} & 53.10\% & 45.97\% & 48.82\% & 54.87\% & 44.92\% & 49.06\% \\ \hline
			\textit{Drug} & 58.99\% & 44.46\% & 48.23\% & 59.77\% & 53.37\% & 55.66\% \\ \hline
			\textit{\textbf{Overall}} & \textbf{59.30\%} & \textbf{51.27\%} & \textbf{54.09\%} & \textbf{61.26\%} & \textbf{53.03\%} & \textbf{56.31\%} \\ \hline
		\end{tabular}
		\label{table:owndict5}
	\end{table}

	\begin{figure}
		\centering
		\includegraphics[width=0.85\linewidth]{images/histogram5}
		\caption{Histogram Perbandingan \textit{F-measure} \textit{Baseline} dengan Eksperimen 2.5}
		\label{fig:owndict5}
	\end{figure}

	\subsubsection{Analisis}
	Dari tabel dan grafik di atas, entitas \textit{drug} mengalami penurunan untuk nilai \textit{precission, recall} dan \textit{f-measure}. Selain itu, entitas \textit{disease} mengalami penurunan pada nilai \textit{precission} tetapi mengalami kenaikan pada nilai \textit{recall} dan \textit{f-measure}. Entitas \textit{symptom} mengalami kenaikan pada nilai \textit{precision} tetapi mengalami penurunan pada nilai \textit{recall} dan \textit{f-measure}. Sedangkan pada entitas \textit{treatment}, terjadi kenaikan nilai \textit{recall} tetapi nilai \textit{precision} dan \textit{f-measure} mengalami penurunan. 
	
	Dari korpus yang \saya~miliki, berikut merupakan informasi statistik dari penggunaan fitur frasa kata kerja:
	\begin{enumerate}
		\item Untuk entitas \textit{disease}, sebanyak 442 entitas merupakan frasa kata benda, 31 entitas merupakan bagian dari frasa kata benda dan 583 entitas bukan merupakan frasa.
		\item Untuk entitas \textit{symptom}, sebanyak 486 entitas merupakan frasa kata benda, 194 entitas merupakan bagian dari frasa kata benda dan 626 entitas bukan merupakan frasa.
		\item Untuk entitas \textit{treatment}, sebanyak 338 entitas merupakan frasa kata benda, 76 entitas merupakan bagian dari frasa kata benda dan 401 entitas bukan merupakan frasa.
		\item Untuk entitas \textit{drug}, sebanyak 152 entitas merupakan frasa kata benda, 8 entitas merupakan bagian dari frasa kata benda dan 194 entitas bukan merupakan frasa.
	\end{enumerate}

	Sedangkan berikut merupakan informasi statistik dari penggunaan fitur frasa kata benda:
	\begin{enumerate}
		\item Untuk entitas \textit{disease}, sebanyak 943 entitas merupakan frasa kata benda, 43 entitas merupakan bagian dari frasa kata benda dan 70 entitas bukan merupakan frasa.
		\item Untuk entitas \textit{symptom}, sebanyak 842 entitas merupakan frasa kata benda, 363 entitas merupakan bagian dari frasa kata benda dan 101 entitas bukan merupakan frasa.
		\item Untuk entitas \textit{treatment}, sebanyak 561 entitas merupakan frasa kata benda, 201 entitas merupakan bagian dari frasa kata benda dan 53 entitas bukan merupakan frasa.
		\item Untuk entitas \textit{drug}, sebanyak 318 entitas merupakan frasa kata benda, 14 entitas merupakan bagian dari frasa kata benda dan 22  entitas bukan merupakan frasa.
	\end{enumerate}
	
	Dari 2 informasi di atas, dapat diambil informasi bahwa sebagian besar entitas \textit{disease} dan \textit{drug} merupakan frasa kata benda dan entitas \textit{symptom} dan \textit{treatment} merupakan frasa kata kerja. Hal ini menjadi seharusnya menjadi informasi pembeda dengan kata yang bukan merupakan entitas. Namun apabila dilihat dari hasil eksperimen ini, performa penggunaan fitur ini tidak terlalu bagus atau bahkan turun di entitas \textit{disease} dan \textit{symptom} tersebut. \Saya~berpendapat hal ini terjadi karena penggunaan fitur ini sudah cukup mewakili informasi fitur POS-Tag, akrena untuk menentukan suatu kata atau kumpulan kata merupakan frasa adalah dengan menggunakan POS-Tag. Selain itu, pada fitur POS-Tag, tidak ada perbedaan antara kata yang merupakan frasa maupun kata yang bukan frasa. Padahal, mayoritas entitas seperti yang telah dijelaskan di atas merupakan frasa. Oleh karena itu, pada sub-eksperimen \ref{eks:subeksminpostag}, penulis menghilangkan fitur POS-Tag dan tetap mempertahankan fitur frasa kata untuk mengetahui hal tersebut. 
	
	%-----------------------------------------------------------------------------%
	\subsection{Eksperimen 2.6: Fitur Kata, Kamus Kesehatan, \textit{Stopword} dan Frasa Kata}\label{eks:subeksminpostag}
	%-----------------------------------------------------------------------------%
	Pada sub-eksperimen ini \saya~menghilangkan fitur POS-Tag berdasarkan hasil dan analisis pada sub-eksperimen \ref{eks:subeksfrasa}.
	
	\subsubsection{Hasil Eksperimen}
	\textbf{Waktu komputasi}: $ 6636.5 $ detik.
	
	Rangkuman hasil sub-eksperimen ini dapat dilihat pada Tabel \ref{table:owndict6} dan Gambar \ref{fig:owndict6}.
	\begin{table}
		\centering
		\caption{Tabel Hasil Eksperimen 2.6 dibandigkan dengan \textit{Baseline}}
		\begin{tabular}{|l|c|c|c|c|c|c|}
			\hline
			\multicolumn{1}{|c|}{\multirow{2}{*}{Entitas}} & \multicolumn{3}{c|}{Baseline (Herwando 2016)} & \multicolumn{3}{c|}{Eksperimen 2.6} \\ \cline{2-7} 
			\multicolumn{1}{|c|}{} & \textit{Precision} & \textit{Recall} & \textit{F-Measure} & \textit{Precision} & \textit{Recall} & \textit{F-Measure} \\ \hline
			\textit{Disease} & 63.68\% & 55.45\% & 59.13\% & 68.67\% & 61.80\% & 64.78\% \\ \hline
			\textit{Symptom} & 61.43\% & 59.21\% & 60.18\% & 63.79\% & 56.10\% & 59.23\% \\ \hline
			\textit{Treatment} & 53.10\% & 45.97\% & 48.82\% & 54.47\% & 46.72\% & 49.58\% \\ \hline
			\textit{Drug} & 58.99\% & 44.46\% & 48.23\% & 60.08\% & 56.70\% & 57.00\% \\ \hline
			\textit{\textbf{Overall}} & \textbf{59.30\%} & \textbf{51.27\%} & \textbf{54.09\%} & \textbf{61.75\%} & \textbf{55.33\%} & \textbf{57.65\%} \\ \hline
		\end{tabular}
		\label{table:owndict6}
	\end{table}
	
	\begin{figure}
		\centering
		\includegraphics[width=0.85\linewidth]{images/histogram6}
		\caption{Histogram Perbandingan \textit{F-measure} \textit{Baseline} dengan Eksperimen 2.4, 2.5 dan 2.6}
		\label{fig:owndict6}
	\end{figure}
	
	\subsubsection{Analisis}
	Dari Tabel \ref{table:owndict6} dan Gambar \ref{fig:owndict6}, terlihat bahwa semua entitas (\textit{disease, symptom, treatment,}) dan \textit{drug} mengalami kenaikan pada nilai \textit{precision, recall,} dan \textit{f-measure}. Seperti yang telah dijelaskan pada sub-eksperimen \ref{eks:subeksfrasa}, penggabungan fitur POS-Tag dan frasa akan memberikan hasil yang lebih rendah. Oleh karena itu, sebaiknya fitur POS-Tag tidak digabung dengan fitur frasa.
	
	Untuk mempermudah dalam membandingkan hasil eksperimen 2.4, 2.5 dan 2.6, \saya~menyajikan grafik \ref{fig:owndict6} untuk membandingkan nilai \textit{F-Measure} pada masing-masing eksperimen. Dapat dilihat bahwa apabila fitur POS-Tag dan Frasa digunakan secara bersama-sama, hasil yang diberikan lebih rendah apabila kedua fitur tersebut dipisah. Namun, apabila dengan menggunakan fitur POS-Tag tanpa Frasa, hasil pada entitas \textit{treatment} dan \textit{drug} lebih bagus. Sedangkan apabila dengan menggunakan fitur Frasa tanpa POS-Tag, hasil pada entitas \textit{disease} dan \textit{symptom} lebih baik. Oleh karena itu \saya~memilih salah satu dari kedua fitur tersebut. Apabila dilihat dari rata-rata \textit{F-Measure}, penggunaan fitur Frasa tanpa POS-Tag memberikan hasil yang paling tinggi ($ 57.65\% $) apabila dibandingkan dengan penggunaan fitur POS-Tag tanpa Frasa ($ 56.99\% $). Oleh karena itu, \saya~mempertahankan fitur Frasa dan tidak menggunakan fitur POS-tag pada eksperimen selanjutnya.
	
	Walaupun pada sub-eksperimen ini hasil yang dicapai lebih baik dari sub-eksperimen sebelumnya, hasilnya tetap lebih rendah dari hasil eksperimen \cite{skripsiKakRadit} pada nilai \textit{recall} dan \textit{f-measure} pada entitas \textit{symptom}. Oleh karena itu \saya~mencoba fitur yang lain, yaitu fitur Kata Sebelum. Untuk penjelasan lebih lanjut akan dibahas pada sub-eksperimen \ref{eks:subekswbef1}.
	
	%-----------------------------------------------------------------------------%
	\subsection{Eksperimen 2.7: Fitur Kata, Kamus Kesehatan, \textit{Stopword}, Frasa Kata dan Kata Sebelum}\label{eks:subekswbef1}
	%-----------------------------------------------------------------------------%
	Pada sub-eksperimen ini \saya~menambahkan fitur baru yaitu fitur 1 kata sebelum. Fitur ini digunakan pada penelitian \cite{skripsiKakRadit} yang juga berkontribusi memberikan hasil terbaik pada penelitiannya. Menurut \saya, ada beberapa entitas yang akan lebih mudah diketahui apabila diketahui kata sebelumnya. Misalnya kata "masuk angin", apabila hanya diberikan informasi kata "angin" tanpa kata "masuk", akan lebih sulit menentukan kata tersebut bagian dari suatu entitas \textit{disease} atau bukan. Oleh karena itu, pada sub-eksperimen ini \saya~mencoba menambahkan fitur tersebut.
	
	\subsubsection{Hasil Eksperimen}
	\textbf{Waktu komputasi}: $ 9275.5 $ detik.
	
	Rangkuman hasil sub-eksperimen ini dapat dilihat pada Tabel \ref{table:owndict7} dan Gambar \ref{fig:owndict7}.
	
	\begin{table}
		\centering
		\caption{Tabel Hasil Eksperimen 2.7 dibandigkan dengan \textit{Baseline}}
		\begin{tabular}{|l|c|c|c|c|c|c|}
			\hline
			\multicolumn{1}{|c|}{\multirow{2}{*}{Entitas}} & \multicolumn{3}{c|}{Baseline (Herwando 2016)} & \multicolumn{3}{c|}{Eksperimen 2.7} \\ \cline{2-7} 
			\multicolumn{1}{|c|}{} & \textit{Precision} & \textit{Recall} & \textit{F-Measure} & \textit{Precision} & \textit{Recall} & \textit{F-Measure} \\ \hline
			\textit{Disease} & 63.68\% & 55.45\% & 59.13\% & 69.49\% & 61.60\% & 64.68\% \\ \hline
			\textit{Symptom} & 61.43\% & 59.21\% & 60.18\% & 64.78\% & 57.15\% & 60.23\% \\ \hline
			\textit{Treatment} & 53.10\% & 45.97\% & 48.82\% & 56.58\% & 44.71\% & 49.54\% \\ \hline
			\textit{Drug} & 58.99\% & 44.46\% & 48.23\% & 62.22\% & 57.28\% & 58.76\% \\ \hline
			\textit{\textbf{Overall}} & \textbf{59.30\%} & \textbf{51.27\%} & \textbf{54.09\%} & \textbf{63.27\%} & \textbf{55.19\%} & \textbf{58.30\%} \\ \hline
		\end{tabular}
		\label{table:owndict7}
	\end{table}
	
	\begin{figure}
		\centering
		\includegraphics[width=0.85\linewidth]{images/histogram7}
		\caption{Histogram Perbandingan \textit{F-measure} \textit{Baseline} dengan Eksperimen 2.7}
		\label{fig:owndict7}
	\end{figure}
	
	\subsubsection{Analisis}
	Melihat pada Tabel \ref{table:owndict7} dan Gambar \ref{fig:owndict7}, dapat diketahui bahwa entitas \textit{disease} dan \textit{treatment} mengalami kenaikan pada nilai \textit{precission}, tetapi mengalami penurunan pada nilai \textit{recall} dan \textit{f-measure}. Sedangkan entitas \textit{symptom} dan \textit{drug} mengalami kenaikan pada nilai \textit{precision, recall,} dan \textit{f-measure}.
	
	Seperti pada penelitian \cite{skripsiKakRadit}, fitur ini berhasil meningkatkan performa dari beberapa entitas, karena fitur ini memberikan informasi tambahan kata sebelumnya, misalnya:
	\begin{itemize}
		\item "penyakit", "penderita", "mengalami" dan "mengalami" dapat memberikan informasi mengenai entitas \textit{disease}
		\item "mengandung", "minum", "pemberian", "obat", "menggunakan" dapat memberikan informasi mengenai entitas \textit{drug}
		\item "dengan", "melakukan", "dilakukan" dapat memberikan informasi mengenai entitas \textit{treatment}
		\item "mengalami", "disertai", "sering", "keluhan", "penyebab" dapat memberikan informasi mengenai entitas \textit{symptom}
	\end{itemize}
	
	Hasil sub-eksperimen ini masih lebih rendah dibandingkan dengan hasil eksperimen \cite{skripsiKakRadit} pada \textit{recall} dan \textit{f-measure} entitas \textit{treatment}. Oleh karena itu, \saya~mencoba menambahkan fitur yang lain yaitu fitur 1 Kata sesudah, yang akan dibahas lebih lanjut pada sub-eksperimen \ref{eks:subekswaf1}.
	
	
	%-----------------------------------------------------------------------------%
	\subsection{Eksperimen 2.8: Fitur Kata, Kamus Kesehatan, \textit{Stopword}, Frasa Kata, Kata Sebelum dan Kata Sesudah}\label{eks:subekswaf1}
	%-----------------------------------------------------------------------------%
	Pada sub-eksperimen ini \saya~menambahkan fitur lain yaitu fitur 1 Kata Setelah. Hal ini karena ada beberapa kasus yang mana apabila suatu kata merupakan sebuah entitas, akan lebih mudah dikenali apabila melihat kata atau konteks setelahnya. Sama seperti contoh pada fitur 1 kata sebelum, misal diberikan kata "masuk angin", apabila hanya diberikan informasi "masuk" tanpa "angin", akan lebih sulit mengenali apakah kata tersebut termasuk entitas \textit{disease} atau bukan. Selain itu, fitur ini juga dapat membedakan kata berentitas dengan kata yang bukan, misalnya kata "masuk angin" dengan "masuk rumah". Apabila informasi pada saat tersebut hanya diberikan kata "masuk" saja tanpa kata setelahnya, akan lebih sulit mengenali kata tersebut termasuk kata berentitas atau bukan.
	
	\subsubsection{Hasil Eksperimen}
	\textbf{Waktu komputasi}: $ 14031.5 $ detik.
	
	Rangkuman hasil sub-eksperimen ini dapat dilihat pada Tabel \ref{table:owndict8} dan Gambar \ref{fig:owndict8}.
	
	\begin{table}
		\centering
		\caption{Tabel Hasil Eksperimen 2.8 dibandigkan dengan \textit{Baseline}}
		\begin{tabular}{|l|c|c|c|c|c|c|}
			\hline
			\multicolumn{1}{|c|}{\multirow{2}{*}{Entitas}} & \multicolumn{3}{c|}{Baseline (Herwando 2016)} & \multicolumn{3}{c|}{Eksperimen 2.7} \\ \cline{2-7} 
			\multicolumn{1}{|c|}{} & \textit{Precision} & \textit{Recall} & \textit{F-Measure} & \textit{Precision} & \textit{Recall} & \textit{F-Measure} \\ \hline
			\textit{Disease} & 63.68\% & 55.45\% & 59.13\% & 70.68\% & 66.18\% & 68.17\% \\ \hline
			\textit{Symptom} & 61.43\% & 59.21\% & 60.18\% & 64.16\% & 59.55\% & 60.23\% \\ \hline
			\textit{Treatment} & 53.10\% & 45.97\% & 48.82\% & 61.02\% & 51.13\% & 54.03\% \\ \hline
			\textit{Drug} & 58.99\% & 44.46\% & 48.23\% & 70.85.\% & 70.33\% & 69.82\% \\ \hline
			\textit{\textbf{Overall}} & \textbf{59.30\%} & \textbf{51.27\%} & \textbf{54.09\%} & \textbf{65.29\%} & \textbf{61.80\%} & \textbf{63.06\%} \\ \hline
		\end{tabular}
		\label{table:owndict8}
	\end{table}
	
	\begin{figure}
		\centering
		\includegraphics[width=0.85\linewidth]{images/histogram8}
		\caption{Histogram Perbandingan \textit{F-measure} \textit{Baseline} dengan Eksperimen 2.8}
		\label{fig:owndict8}
	\end{figure}
	
	\subsubsection{Analisis}
	Melihat pada Tabel \ref{table:owndict8} dan Gambar \ref{fig:owndict8}, dapat diketahui bahwa hanya entitas \textit{symptom} yang mengalami penurunan nilai pada \textit{precision}, tetapi nilai \textit{recall} dan \textit{f-measure}-nya naik. Sedangkan entitas lain mengalami kenaikan pada nilai \textit{precision, recall} dan \textit{f-measure}. Oleh karena itu, setelah \saya~mencoba kemungkinan fitur yang memberikan kontribusi dalam penelitian ini, \saya~mencoba arsitektur untuk model RNNs yang lain. Penjelasan lebih lanjut akan dibahas pada eksperimen \ref{eks:eks2}.
	
	
	
  %-----------------------------------------------------------------------------%
  
  
\section{Skenario 3: Skenario Pengujian Arsitektur RNNs}\label{eks:eks2}
    
    Pada eksperimen ini, \saya~mencoba dua buah arsitektur RNNs yang telah \saya~usulkan pada Bab 3 yaitu RNNs dengan 1 layer dan RNNs dengan 2 layer. Fitur yang digunakan dalam pengujian ini yaitu kombinasi fitur yang menghasilkan akurasi terbaik pada eksperimen pertama, yaitu fitur kata itu sendiri, kamus kesehatan, \textit{stop word}, frasa kata, 1 kata sebelum dan 1 kata sesudah.
    
    \subsection{Eksperimen 3.1: Menguji Arsitektur LSTMs 1 Layer}\label{eks2:subeksrnn1}
    %-----------------------------------------------------------------------------%
    Pada sub-eksperimen ini, \saya~menggunakan struktur RNNs yang mana semua fitur digabung menjadi satu dalam sebuah \textit{timestep}.
    Artinya fitur-fitur yang berbeda tersebut akan digabung atau di-\textit{concat} menjadi sebuah vektor yang akan menjadi \textit{input} bagi LSTMs ini. LSTMs inilah yang digunakan pada eksperimen pertama, sehingga hasilnya sama dengan sub-eksperimen \ref{eks:subekswaf1}.
    
    \subsubsection{Hasil Eksperimen}
    \textbf{Waktu komputasi}: $ 14031.5 $ detik.
    
    \begin{table}
    	\centering
    	\caption{Tabel Hasil Eksperimen 3.1 dibandigkan dengan \textit{Baseline}}
    	\begin{tabular}{|l|c|c|c|c|c|c|}
    		\hline
    		\multicolumn{1}{|c|}{\multirow{2}{*}{Entitas}} & \multicolumn{3}{c|}{Baseline (Herwando 2016)} & \multicolumn{3}{c|}{Eksperimen 2.7} \\ \cline{2-7} 
    		\multicolumn{1}{|c|}{} & \textit{Precision} & \textit{Recall} & \textit{F-Measure} & \textit{Precision} & \textit{Recall} & \textit{F-Measure} \\ \hline
    		\textit{Disease} & 63.68\% & 55.45\% & 59.13\% & 70.68\% & 66.18\% & 68.17\% \\ \hline
    		\textit{Symptom} & 61.43\% & 59.21\% & 60.18\% & 64.16\% & 59.55\% & 60.23\% \\ \hline
    		\textit{Treatment} & 53.10\% & 45.97\% & 48.82\% & 61.02\% & 51.13\% & 54.03\% \\ \hline
    		\textit{Drug} & 58.99\% & 44.46\% & 48.23\% & 70.85.\% & 70.33\% & 69.82\% \\ \hline
    		\textit{\textbf{Overall}} & \textbf{59.30\%} & \textbf{51.27\%} & \textbf{54.09\%} & \textbf{65.29\%} & \textbf{61.80\%} & \textbf{63.06\%} \\ \hline
    	\end{tabular}
    	\label{table:owndict9}
    \end{table}
    
    \begin{figure}
    	\centering
    	\includegraphics[width=0.85\linewidth]{images/histogram9}
    	\caption{Histogram Perbandingan \textit{F-measure} \textit{Baseline} dengan Eksperimen 3.1}
    	\label{fig:owndict9}
    \end{figure}
    
    \subsubsection{Analisis}
    Pada eksperimen ini, hasil yang sudah lebih baik apabila dibandingkan dengan hasil yang dicapai \cite{skripsiKakRadit} di entitas. Namun, dari eksperimen sebelumnya, terdapat akurasi yang turun, yaitu nilai \textit{precision} untuk entitas \textit{symptom}. Menurut \saya~hal ini terjadi karena informasi fitur yang berbeda-beda dijadikan satu, sehingga ada kemungkinan hilangnya informasi dari masing-masing fitur tersebut. Oleh karena itu, untuk mengatasi permasalahan tersebut \saya~mengusulkan arsitektur yang mana masing-masng kelompok fitur yang berbeda dipisahkan dan menjadi \textit{input} bagi masing-masing LSTMs. Untuk penjelasan eksperimen ini akan dijelaskan pada sub-eksperimen \ref{eks2:subeksrnn2}
    
    
    \subsection{Eksperimen 3.2: Menguji Arsitektur LSTMs 2 Layer Multi-Input}\label{eks2:subeksrnn2}
    %-----------------------------------------------------------------------------%
    Pada sub-eksperimen sebelumnya, fitur-fitur yang berbeda digabung menjadi satu, sehingga ada kemungkinan hilangnya informasi dari fitur tersebut. Oleh karena itu, \saya~mengusulkan adanya layer tambahan setelah masing-masing fitur tersebut masuk ke dalam model. \Saya~mengusulkan bahwa masing-masing kelompok fitur menjadi \textit{input} LSTMs secara terpisah. Setelah masuk di RNNs, \textit{output} dari masing-masing LSTMs tersebut di-\textit{merge} ke dalam sebuah layer, lalu masuk kembali ke LSTMs untuk melihat konteks fitur-fitur sebelumnya. Dengan diusulkannya arsitektur RNNs ini \saya~berharap bahwa masing-masing fitur terjaga informasinya dan tidak terganggu dengan informasi lain.
    
    \subsubsection{Hasil Eksperimen}
    \textbf{Waktu komputasi}: $ 20362.5 $ detik.
    
    Rangkuman hasil sub-eksperimen ini dapat dilihat pada Tabel \ref{table:owndict10} dan Gambar \ref{fig:owndict10}.
        
    \begin{table}
    	\centering
    	\caption{Tabel Hasil Eksperimen 3.2 dibandigkan dengan \textit{Baseline}}
    	\begin{tabular}{|c|c|c|c|c|}
    		\hline
    		& \textit{Precision} & \f{\f{Recall}} & \f{\f{F-Measure}} \\ \hline
    		\textit{Disease}      & 67.47\%             & 67.19\%        & 66.31\%           \\ \hline
    		\textit{Symptom}      & 64.90\%             & 60.63\%        & 62.13\%           \\ \hline
    		\textit{Treatment}    & 63.92\%             & 53.13\%        & 56.51\%           \\ \hline
    		\textit{Drug}		  & 66.39\%             & 62.33\%        & 63.61\%           \\ \hline
    		\textit{\textbf{Overall}}&\textbf{65.67\%}  & \textbf{60.82\%}& \textbf{62.14\%} \\ \hline
    	\end{tabular}
    	\label{table:owndict10}
    \end{table}
    
    \begin{figure}
    	\centering
    	\includegraphics[width=0.85\linewidth]{images/histogramrnnv2}
    	\caption{Histogram Perbandingan \textit{F-measure} \textit{Baseline} dengan Eksperimen 3.1, dan 3.2}
    	\label{fig:owndict10}
    \end{figure}

	\subsubsection{Analisis}
	Pada eksperimen ini, hasil yang sudah lebih baik apabila dibandingkan dengan hasil yang dicapai \cite{skripsiKakRadit} di masing-masing entitas dan lebih baik dibandingkan eksperimem \ref{eks2:subeksrnn1} pada identifikasi entitas \textit{symptom} dan \textit{treatment}.
