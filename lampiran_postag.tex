%!TEX root = skripsi.tex

\begin{longtable}{|c|c|p{.50\textwidth}|p{.40\textwidth}|}
	\caption{Daftar POSTAG Bahasa Indonesia}\label{lampiran:postag}
	% header and footer information
	\hline
	\multicolumn{1}{|c|}{\textbf{No}} & \multicolumn{1}{c|}{\textbf{Tag}} & \multicolumn{1}{c|}{\textbf{Deskripsi}} & \multicolumn{1}{c|}{Contoh} \\
	\hline
	\endhead
	\hline
	\endfoot
	1 & CC & Konjungtor koordinatif menghubungkan dua satuan bahasa atau lebih yang sederajat (kata dengan kata; frasa dengan frasa; atau klausa dengan klausa) yang masing-masing mempunyai kedudukan yang setara dalam struktur kalimat. & dan;tetapi;atau \\ \hline
	2 & CD & Numeralia kardinal; yaitu numeralia yang menjadi jawaban atas pertanyaan "Berapa?" & dua; juta; enam; 7916; sepertiga; 0;025; 0;525; banyak; kedua; ribuan; 2007; 25 \\ \hline
	3 & OD & Numeralia ordinal menyatakan urutan dan menjadi jawaban atas pertanyaan "Yang keberapa?" & ketiga; ke-4; pertama \\ \hline
	4 & DT & Artikel bertugas membatasi makna nomina. & para; sang; si \\ \hline
	5 & FW & Kata bahasa asing adalah kata yang berasal dari bahasa asing yang belum diserap ke dalam bahasa Indonesia. Pada dasarnya; kata bahasa asing adalah katayang tidak terdapat di dalam kamus bahasa Indonesia. &  \\ \hline
	6 & IN & Preposisi menghubungkan kata atau frasa dengan konstituen di depan preposisi tersebut sehingga terbentuk frasa preposisional. & dalam; dengan; di; ke; oleh; pada; untuk \\ \hline
	7 & JJ & Adjektiva adalah kata yang memberikan keterangan yang lebih khusus tentang sesuatu yang dinyatakan oleh nomina dalam kalimat. & bersih; panjang; hitam; lama; jauh; marah; suram; nasional; bulat \\ \hline
	8 & MD & Verba modal dan verba bantu. & boleh; harus; sudah; mesti; perlu \\ \hline
	9 & NEG & Kata ingkar. & tidak; belum; jangan \\ \hline
	10 & NN & Nomina; yaitu kata yang mengacu pada manusia; binatang; benda; konsep; atau pengertian & monyet; bawah; sekarang; rupiah \\ \hline
	11 & NNP & Proper noun adalah nama spesifik dari seseorang; sesuatu; atau sebuah tempat. & Boediono; Laut Jawa; Indonesia; India; Malaysia; Bank Mandiri; BBKP; Januari; Senin; Idul Fitri; Piala Dunia; Liga Primer; Lord of the Rings: The Return of the King \\ \hline
	12 & NND & Penggolong menempatkan nomina ke dalam sebuah kelompok tertentu dalam jumlah tertentu; misalnya orang dalam dua orang prajurit. & orang; ton; helai; lembar \\ \hline
	13 & PR & Demonstrativa atau pronomina penunjuk. & ini; itu; sini; situ \\ \hline
	14 & PRP & Pronomina persona; yaitu pronomina yang dipakai untuk mengacu pada orang. & saya; kami; kita; kamu; kalian; dia; mereka \\ \hline
	15 & RB & Adverbia; atau disebut juga kata keterangan. & sangat; hanya; justru; niscaya; segera \\ \hline
	16 & RP & Dalam penelitian ini; POS tag RP menandai partikel penegas; yaitu partikel yang digunakan untuk menegaskan kalimat interogatif; imperatif; atau deklaratif. & pun; -lah; -kah \\ \hline
	17 & SC & Konjungtor subordinatif menghubungkan dua klausa atau lebih dan salah satu dari klausa-klausa tersebut merupakan klausa subordinatif. & sejak; jika; seandainya; supaya; meski; seolah-olah; sebab; maka; tanpa; dengan; bahwa; yang; lebih; daripada; semoga \\ \hline
	18 & SYM & Simbol; yang diberi POS tag SYM; meliputi simbol matematis; misalnya +; dan simbol mata uang; misalnya IDR &  \\ \hline
	19 & UH & Interjeksi mengungkapkan rasa hati atau perasaan pembicara dan secara sintaktis tidak berhubungan dengan kata-kata lain di dalam kalimat atau ujaran. & brengsek; oh; ooh; aduh; ayo; mari; hai \\ \hline
	20 & VB & Verba; yang diberi POS tag VB; dapat berupa verba transitif; verba intransitif; verba aktif; verba pasif; dan kopula. & merancang; mengatur; pergi; bekerja; tertidur \\ \hline
	21 & WH & Pronomina penanya digunakan dalam kalimat interogatif sebagai pemarkah pertanyaan. & siapa; apa; mana; kenapa; kapan; di mana; bagaimana; berapa \\ \hline
	22 & X & Kata atau bagian dari kalimat yang tidak diketahui atau belum diketahui secara pasti kategorinya. & statement \\ \hline
\end{longtable}