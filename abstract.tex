%!TEX root = skripsi.tex
%
% Halaman Abstract
%
% @author  Andreas Febrian
% @version 1.00
%

\chapter*{ABSTRACT}

\vspace*{0.2cm}

\noindent \begin{tabular}{l l p{11.0cm}}
	Name&: & \penulis \\
	Program&: & \programEng \\
	Title&: & \judulInggris \\
\end{tabular} \\ 

\vspace*{0.5cm}

\noindent 

Semantic Role Labeling (SRL) has been extensively studied, mostly for understanding English formal language. However, only a few reports exist for informal conversational language, especially for language being used in the chatbot system. The challenges of informal texting language include a wide variety of slangs and abbreviations, short sentences, as well as disorganized grammars. In this work, we propose a new set of semantic roles and a Context-Aware Bi-Directional Long Short-Term Memory Networks model for solving SRL task on informal conversational language. We utilized word embedding and linguistic components as our main features. The SRL task was mainly evaluated on Indonesian informal conversational language used on chatting platform. Although this is a pilot task, we obtained a really promising result with F1 score of 74.78\%.


\vspace*{0.2cm}

\noindent Keywords: \\ 
\noindent \mer, Semantic Role Labeling, deep learning, conversational language, RNNs \\ 

\newpage