%!TEX root = skripsi.tex
%
% Halaman Abstract
%
% @author  Andreas Febrian
% @version 1.00
%

	\chapter*{ABSTRACT}

\vspace*{0.2cm}

\noindent \begin{tabular}{l l p{11.0cm}}
	Name&: & \penulis \\
	Program&: & \programEng \\
	Title&: & \judulInggris \\
\end{tabular} \\ 

\vspace*{0.5cm}

\noindent 
Many Moslems want to recite \quran. Unfortunatelly, someone who is reciting \quran~needs a partner to help evaluating the recitation. To help that process, this research develops a system that is able to automatically evaluate \quran~recitation. The system uses \f{mel frequency cepstral coefficient} (MFCC) and \f{shifted delta cepstral coefficient} (SDCC) feature, with \f{support vector machine} (SVM) and \f{Gaussian mixture model} (GMM) classification method. The experiment is applied to every ayah in juz 30 of \quran, and the result shows that the best combination to use in the system is SDCC feature with GMM classification method. \\

% \quran~adalah kitab suci bagi umat Muslim. Di dalam \quran~terkandung berbagai macam hakikat kehidupan. Banyak umat Muslim yang ingin belajar membaca \quran~serta menghafalkannya. Orang yang belajar membaca atau menghafalkan \quran~membutuhkan rekan yang mengerti cara membacanya, agar dapat membantu mengevaluasi bacaan \quran. Namun rekan tersebut tidak setiap waktu dapat membantu mengevaluasi. Oleh karena itu, penelitian ini dilakukan dalam rangka membangun sistem pengenalan suara yang mampu mengevaluasi pembacaan \quran~secara otomatis, guna membantu proses evaluasi tersebut. Sistem yang dikembangkan dalam penelitian ini mengekstrak fitur \f{mel frequency cepstral coefficient} (MFCC) dan \f{shifted delta cepstral coefficient} (SDCC) dari audio. Fitur tersebut diklasifikasi menggunakan metode klasifikasi \f{support vector machine} (SVM) dan \f{Gaussian mixture model} GMM. Eksperimen dilakukan terhadap ayat-ayat di juz 30 \quran. Hasil eksperimen dalam penelitian ini menunjukkan bahwa kombinasi yang paling tepat untuk digunakan dalam sistem ini adalah penggunaan fitur SDCC dan metode klasifikasi GMM. \\

% \\ Quran is a holy book for Moslems. It contains much nature in this life. Therefore many of Moslems wants to memorize quran. Quran is originally delivered in Arabic, and it is memorized in Arabic also. The are some ways to memorize quran. One of them is to get help with the second person for heeding the recitation. Whenever the recitation of the first person is wrong, then the second person gives a sign to the first person. Unfortunatelly it cannot be done if there is no second person. Another way is to recite some ayahs, then match them to the original text in quran. That way may slow down memorizing process and also it gives a chance to the reciter to see the next part of quran that should not be seen by him. To make the memorizing process more practical, then a computer system that can cover the second person to match quran recitation to the original quran text, is developed. This system uses Automatic Speech Recognition (ASR).

\vspace*{0.2cm}

\noindent Keywords: \\ 
\noindent \quran, evaluation, MFCC, SDCC, SVM, GMM \\

\newpage