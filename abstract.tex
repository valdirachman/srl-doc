%!TEX root = skripsi.tex
%
% Halaman Abstract
%
% @author  Andreas Febrian
% @version 1.00
%

\chapter*{ABSTRACT}

\vspace*{0.2cm}

\noindent \begin{tabular}{l l p{11.0cm}}
	Name&: & \penulis \\
	Program&: & \programEng \\
	Title&: & \judulInggris \\
\end{tabular} \\ 

\vspace*{0.5cm}

\noindent 

Semantic Role Labeling (SRL) has been extensively studied, mostly for understanding English formal language. However, only a few reports exist for informal conversational language, especially for language being used in the chatbot system. In Indonesian, both formal and conversational language are barely tapped for building SRL system. In this work, we focus on solving SRL on Indonesian conversational language. Our contributions include introducing a new set of semantic roles and proposing an attention mechanism on top of Long Short-Term Memory Networks architecture. The challenges of Indonesian conversational language include a wide variety of slangs and abbreviations, short sentences, as well as disorganized grammars. Although this is a pilot task, we obtained a really promising result with F1 score of 82.68\%.

\vspace*{0.2cm}

\noindent Keywords: \\ 
\noindent Semantic Role Labeling, deep learning, conversational language, RNNs \\ 

\newpage