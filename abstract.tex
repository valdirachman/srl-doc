%!TEX root = skripsi.tex
%
% Halaman Abstract
%
% @author  Andreas Febrian
% @version 1.00
%

\chapter*{ABSTRACT}

\vspace*{0.2cm}

\noindent \begin{tabular}{l l p{11.0cm}}
	Name&: & \penulis \\
	Program&: & \programEng \\
	Title&: & \judulInggris \\
\end{tabular} \\ 

\vspace*{0.5cm}

\noindent 

Nowadays, anyone can use online health forum to look for information of disease without meet the doctor directly. Through the forum, someone just need to post his symptom and question to the form given. A lot of information which can be retrieved from the forum, such as symptom, disease, or treatment. On this research, we try to extract disease, symptom, treatment and drug automatically. We see this problem as a sequence labeling problem so we propose Deep Learning technique using Recurrent Neural Networks (RNNs), because RNNs is state-of-the-art for sequence labeling problem. We propose some features such as word features, medical dictionary, stop word, POS-Tag, word phrase(verb and noun), word before and word after. Furthermore we propose two RNNs architectures, which are LSTMs in one layer and LSTMs in two layer with multi-input. The result of this experiment shows that the model proposed can give the good result. Based on the experiment using the combination features of its own word, medical dictionary, stop word, word phrase (noun and verb), 1 word before and 1 word after using the first RNNs architecture achieve f-measure $ 63.06\% $, and using second RNNs architecture achieve f-measure $ 62.14\% $. Thats result is better than the baseline used \citep{skripsiKakRadit}, f-measure $ 54.09\% $.


\vspace*{0.2cm}

\noindent Keywords: \\ 
\noindent \mer, RNNs, disease, symptom, treatment, drug \\ 

\newpage