%!TEX root = skripsi.tex
%
% Halaman Abstract
%
% @author  Andreas Febrian
% @version 1.00
%

\chapter*{ABSTRACT}

\vspace*{0.2cm}

\noindent \begin{tabular}{l l p{11.0cm}}
	Name&: & \penulis \\
	Program&: & \programEng \\
	Title&: & \judulInggris \\
\end{tabular} \\ 

\vspace*{0.5cm}

\noindent 

Nowadays, everyone can use online health forums for seeking information regarding health issues without directly visiting a doctor in the hospital. There are thousands of health-related posts generated by thousands of online users everyday. As a result, A huge amount of information can be extracted from such online forums. In this work, we focus on seeking a good computational model to automatically extract the names of disease, symptom, treatment, and drug due to its usefulness for future task, such as Medical Question Answering systems. Furthermore, we formulate our problem as a sequence labeling problem using machine learning approach. We propose Deep Learning technique using Recurrent Neural Networks (RNNs), because RNNs is state-of-the-art for sequence labeling problem. We propose some features such as word features, medical dictionary, stop word, POS-Tag, word phrase(verb and noun), word before and word after. Furthermore we propose two RNNs architectures, which are LSTMs in one layer and LSTMs in two layer with multi-input. The result of this experiment shows that the model proposed can give the good result. Based on the experiment using the combination features of its own word, medical dictionary, stop word, word phrase (noun and verb), 1 word before and 1 word after using the first RNNs architecture achieve f-measure $ 63.06\% $, and using second RNNs architecture achieve f-measure $ 62.14\% $. Thats result is better than the baseline used \citep{skripsiKakRadit}, f-measure $ 54.09\% $.


\vspace*{0.2cm}

\noindent Keywords: \\ 
\noindent \mer, RNNs, disease, symptom, treatment, drug \\ 

\newpage