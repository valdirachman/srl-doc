%!TEX root = skripsi.tex
%-----------------------------------------------------------------------------%
\chapter{\babEmpat} \label{eksperimen}
%-----------------------------------------------------------------------------%

Bab ini akan membahas mengenai implementasi pada penelitian yang terdiri atas tahap pengumpulan data, pra-pemrosesan, pengembangan model, eksperimen dan evaluasi. Setiap fitur yang \saya~usulkan pada Bab 3 juga akan dijelaskan langkah pengimplementasian pada bab ini.

\section{Pengumpulan Data}
\Saya~melakukan pengumpulan data dengan menggunakan ide implementasi dari \cite{skripsiKakRadit} yang kemudian \saya~modifikasi sesuai dengan kebutuhan. Bahasa program yang \saya~gunakan untuk melakukan pengumpulan data ini adalah Java, dengan menggunakan library JSoup untuk mengunduh isi forum sebuah situs. Hasil dari pengumpulan data ini \saya~gabungkan dengan data penelitian milik \cite{skripsiKakRadit}.

\begin{kode}\label{(kode_crawling)}
	\SetKwInOut{Input}{Input}
	\SetKwInOut{Output}{Output}
	
	\SetKwProg{Fn}{Function}{ is}{end}
	\Fn{downloadPage(link)}{
		\Input{link of an online health forum}
		\Output{content of forum}
		\BlankLine
		
		sql = selectFromDB(link)\;
		res = execOnDB(sql)\;
		\BlankLine
		
		\If{res != empty}{
			insertToDB(sql)\;
			doc = JSoup.connect(link)\;
			writeToFile(doc.getJudulKeluhan())\;
			writeToFile(doc.getIsiKeluhan())\;
			writeToFile(doc.getJawaban)\;
		}
	}
	
	\caption{\textit{Pseudocode} untuk melakukan pengumpulan data}	
\end{kode}

Hasil dari pengumpulan data ini yaitu \saya~mendapatkan 2065 \textit{post} dari forum kesehatan \textit{online} pada situs \textit{www.tanyadok.com}.

\section{Pra-Pemrosesan}
Tahap selanjutnya yaitu tahap pra-pemrosesan. Seperti yang telah dijelaskan pada bab metodologi, \saya~melakukan tiga buah pekerjaan di tahap ini, yaitu melakukan pembersihan data, tokenisasi dan pemotongan kalimat. Berikut merupakan penjelasan dari masing-masing pekerjaan tersebut:

\subsection{Pembersihan Data}
Tahap pembersihan data bertujuan untuk menghilangkan karakter yang bukan merupakan ASCII. Hal ini \saya~lakukan supaya dalam tahap ekstraksi fitur POS Tagging tidak memiliki masalah karena terdapat karakter bukan ASCII. Selain itu, di dalam dokumen terdapat banyak \textit{email} dan \textit{url} yang unik sehingga mengakibatkan sistem akan menganggap token-token tersebut merupakan token yang unik dan berbeda. Untuk menangani hal tersebut \saya~melakukan normalisasi dengan mengubah semua token \textit{email} dan \textit{url} menjadi kata "email" dan "url" sehingga tetap mempertahankan keberadaan kedua token tersebut. Selain itu \saya~juga mengganti beberapa karakter yang bukan alfanumerik menjadi beberapa token dalam representasi kata, seperti karakter "\&" menjadi "dan", "\textless" dan "\textgreater" menjadi token "kurang dari" dan "lebih dari". Hal ini \saya~lakukan karena korpus yang \saya~gunakan dalam bentuk berkas \textit{xml} yang tidak mengizinkan adanya ketiga karakter tersebut. Kemudian \saya~juga mengubah karakter "/" menjadi "atau" supaya mudah dalam ekstraksi fitur kata itu sendiri dengan menggunakan \textit{word embedding}. \ref{kode:cleaning} merupakan \textit{pseudocode} untuk melakukan pembersihan data yang \saya~gunakan.


\begin{kode}
	\label{kode:cleaning}
	\SetKwInOut{Input}{Input}
	\SetKwInOut{Output}{Output}
	
	\SetKwProg{Fn}{Function}{ is}{end}
	\Fn{downloadPage(sentence)}{
		\Input{sentence before cleaning}
		\Output{sentence which has cleaned}
		\BlankLine
		
		sentence.removeByRegex(non-ASCII regex)\;
		sentence.replace(email-regex, "email")\;
		sentence.replace(url-regex, "url")\;
		sentence.replace(\&, "dan")\;
		sentence.replace(\textless, "kurang dari")\;
		sentence.replace(\textgreater, "lebih dari")\;
		sentence.replace(/, "atau")\;
		\BlankLine
		
		\Return sentence;
	}
	
	\caption{\textit{Pseudocode} untuk melakukan pembersihan data}	
\end{kode}

\subsection{Tokenisasi}
Seperti yang dijelaskan pada \ref{bab:tiga}, pada tahap tokenisasi \saya~melakukan pemisahan antar kata dan antar token yang berbeda jenis, seperti token alfabet dengan numerik, alfanumerik dengan non-alfanumerik dan menghilangkan karakter spasi yang berlebih. Dalam mengimplementasikan tahap ini, \saya~menggunakan bahasa Ruby. Berikut merupakan \textit{pseudocode} untuk melakukan tokenisasi.

\begin{kode}
	\label{kode:tokenisasi}
	\SetKwInOut{Input}{Input}
	\SetKwInOut{Output}{Output}
	
	\SetKwProg{Fn}{Function}{ is}{end}
	\Fn{tokenization(sentence)}{
		\Input{sentence before tokenization}
		\Output{sentence which has tokenized}
		\BlankLine
		
		sentence.replaceByRegex([alfabet][numerik], [alfabet] [numerik])\;
		sentence.replaceByRegex([numerik][alfabet], [numerik] [alfabet])\;
		sentence.replaceByRegex([alfanumerik][non-alfanumerik], [alfanumerik] [non-alfanumerik])\;
		sentence.replaceByRegex([non-alfanumerik][alfanumerik], [non-alfanumerik] [alfanumerik])\;
		sentence.replaceByRegex([\textbackslash s]+, " ")\;
		\BlankLine
		
		\Return sentence;
	}
	\caption{\textit{Pseudocode} untuk melakukan pembersihan data}	
\end{kode}

\subsection{Pemotongan Kalimat}
Implementasi yang \saya~lakukan tahap ini bertujuan untuk mendapatkan sebuah \textit{instance} sebagai \textit{input} dari program RNNs di tahap eksperimen. Pemotongan dilakukan pada masing-masing \textit{post}. Pada pemotongan kalimat ini, penulis menerapkan aturan berbeda yang telah dijelaskan pada bab 3 karena jumlah kata pada sebuah kalimat yang dipisahkan dengan tanda baca ".", "!" dan "?" sangat jauh berbeda. Dengan implementasi pemotongan kalimat ini, \saya~berupaya untuk menghindari kasus kalimat yang \textit{sparse}, yaitu adanya kalimat yang memiliki jumlah token sangat renggang. Berikut merupakan implementasi dari tahap ini.

\begin{kode}
	\label{kode:potong}
	\SetKwInOut{Input}{Input}
	\SetKwInOut{Output}{Output}
	
	\SetKwProg{Fn}{Function}{ is}{end}
	\Fn{sentenceSplitting(post, limit)}{
		\Input{post, minimal limit number of word in a sentence}
		\Output{array of sentence}
		\BlankLine
		
		arrSentence = post.splitByRegex([?\.!,])\;
		temp = []\;
		arrResult = []\;
		\ForEach{sentence in arrSentence}{
			\uIf{len(temp) > limit}{
				arrResult.append(temp)\;
				temp = []\;
			}
			\Else{
				temp += sentence
			}
		}
		\BlankLine
		
		\Return arrResult;
	}
	
	\caption{\textit{Pseudocode} untuk melakukan pembersihan data}	
\end{kode}

\section{Pelabelan}
Pada tahap ini \saya~melakukan pelabelan pada data baru yang telah diunduh. Sebelumnya, \cite{skripsiKakRadit} telah melabeli 200 buah \textit{post} dan pada penelitian ini \saya~melakukan pelabelan terhadap 109 buah \textit{post} yang \saya~pilih dari hasil pengumpulan data. \Saya~melakukan pemilihan berdasarkan banyaknya kalimat dalam sebuah \textit{post}. Untuk aturan pelabelan, \saya~mengikuti atuan pelabelan yang dilakukan oleh \cite{skripsiKakRadit} dalam penelitiannya. Pelabelan ini dilakukan selama 2 minggu.

\section{Pengembangan Model}
\subsection{Ekstraksi Fitur}
Ekstraksi fitur dilakukan dengan menggunakan program yang diimplementasikan dalam bahasa Python. Keluaran dari ekstraksi fitur ini adalah vektor kata untuk masing-masing kalimat yang disimpan dalam format JSON. Masing-masing kalimat dalam sebuah \textit{post} disimpan dalam sebuah \textit{array} yang kemudian keseluruhan \textit{post} disimpan dalam \textit{hash} dengan indeks yang telah didefinisikan pada saat tahap pengumpulan data.

\subsubsection{Fitur Kata Itu Sendiri}
Dalam melakukan ekstraksi fitur kata itu sendiri, \saya~menggunakan \textit{library} gensim yang disediakan secara gratis. Gensim mengimplementasikan \textit{word embedding} melalui \textit{library} bernama word2vec. Sebelum melakukan ekstraksi fitur, \saya~melakukan \textit{training} model \textit{word embedding} dengan data yang \saya~unduh dari berbagai artikel kesehatan di beberapa situs. Setelah model didapatkan, \saya~melakukan ekstraksi dari masing-masing kata pada korpus.

\begin{kode}
	\label{kode:ekstraksiownword}
	\SetKwInOut{Input}{Input}
	\SetKwInOut{Output}{Output}
	\SetKwProg{Fn}{Function}{ is}{end}
	\Fn{wordToVector(model, arrWord)}{
		\Input{model word embedding, array of word in a sentence}
		\Output{array of word vector}
	
		\BlankLine
		arrVector = []\;
		\ForEach{word in arrWord}{
			arrVector.append(model.getVector(word))
		}
		
		\BlankLine
		\Return arrVector;
	}
	
	\caption{\textit{Pseudocode} untuk melakukan ekstraksi fitur kata itu sendiri}	
\end{kode}

\subsubsection{Ekstraksi Fitur Part of Speech Tag}
Dalam mengimplementasikan ekstraksi fitur POS Tag, \saya~menggunakan \textit{tools} Stanford POS Tagger dan model POS \textit{tagger} yang dikembangkan oleh \cite{dinakaramani2014designing}. Pertama-tama \saya~melakukan pemberian tag pada setiap kaliman di dalam korpus, kemudian mengubah hasil tag tersebut menjadi bentuk \textit{one-hot-vetor}. Berikut merupakan \textit{pseudocode} dalam melakukan ekstraksi fitur POS Tag untuk sebuah kalimat yang \saya~lakukan.

\begin{kode}
	\label{kode:ekstraksipostag}
	\SetKwInOut{Input}{Input}
	\SetKwInOut{Output}{Output}
	
	\SetKwProg{Fn}{Function}{ is}{end}
	\Fn{posTagExtract(model, sentence)}{
		\Input{tagger of POS Tag,sentence}
		\Output{array of one hot vector}
		\BlankLine
		
		sentenceTagged = model.tagSentence(sentence)\;
		tagOnly = getTagOnly(sentenceTagged)\;
		
		\BlankLine
		posTagFeature = []\;
		\ForEach{tag in tagOnly}{
			posTagFeature.append(tag.convertToOneHotVector())
		}
		\BlankLine
			
		\Return posTagFeature;
	}
	
	\caption{\textit{Pseudocode} untuk melakukan ekstraksi fitur kata itu sendiri}	
\end{kode}

\subsubsection{Ekstraksi Fitur Stop Word}
Ekstraksi fitur \textit{stop word} \saya~lakukan dengan menggunakan kamus \textit{stop word} yang digunakan oleh Taufik (2015) dalam melakukan pengenalan entitas bernama. Setiap kata yang merupakan \textit{stop word} memiliki nilai fitur [ 0.0, 1.0 ] dan kata yang bukan merupakan \textit{stop word} memiliki nilai fitur [ 1.0, 0.0 ]. \Saya~menggunakan bahasa pemrograman Pyhton dalam mengimplementasikan ekstraksi fitur ini.

\begin{kode}
	\label{kode:ekstraksistopword}
	\SetKwInOut{Input}{Input}
	\SetKwInOut{Output}{Output}
	
	\SetKwProg{Fn}{Function}{ is}{end}
	\Fn{stopWordExtract(dictionary, sentence)}{
		\Input{dictionary of stop word,sentence}
		\Output{array of one hot vector}
		
		\BlankLine
		stopWordFeature = []\;
		\ForEach{word in tagOnly}{
			posTagFeature.append(dictionary.isExist(word))
		}
		\BlankLine
		
		\Return stopWordFeature;
	}
	
	\caption{\textit{Pseudocode} untuk melakukan ekstraksi fitur \textit{stop word}}	
\end{kode}

\subsubsection{Ekstraksi Fitur Kamus Kesehatan}
Pada dasarnya implementasi ekstraksi fitur kamus kesehatan mirip dengan implementasi ekstraksi fitur \textit{stop word}. Perbedaanya yaitu pada penggunaan \textit{resource}, yang mana ekstraksi fitur \textit{stop words} \saya~lakukan dengan menggunakan kamus \textit{stop word}, sedangkan pada fitur ini \saya~menggunakan kamus kesehatan. Kamus kesehatan yang saya gunakan sama dengan kamus pada penelitian \cite{skripsiKakRadit}, yang mana terdapat 4 kamus, yaitu kamus \textit{disease}, \textit{symptom}, \textit{treatment} dan \textit{drug}. Setiap kata yang terdaftar di dalam kamus kesehatan memiliki nilai fitur [ 0.0, 1.0 ] dan kata yang bukan merupakan \textit{stop word} memiliki nilai fitur [ 1.0, 0.0 ]. \Saya~menggunakan bahasa pemrograman Pyhton dalam mengimplementasikan ekstraksi fitur ini.

\begin{kode}
	\label{kode:ekstraksistopword}
	\SetKwInOut{Input}{Input}
	\SetKwInOut{Output}{Output}
	
	\SetAlgoLined
	\SetKwProg{Fn}{Function}{ is}{end}
	\Fn{dictExtract(dictionary, sentence)}{
		\Input{dictionary of stop word,sentence}
		\Output{array of one hot vector}
		\BlankLine
		
		dictFeature = []\;
		\ForEach{word in sentence}{
			dictFeature.append(dictionary.isExist(word))
		}
		\BlankLine
		
		\Return dictFeature;
	
	}
	\BlankLine
	
	\Fn{dictExtractAll(sentence)}{
		\Input{dictionary of stop word,sentence}
		\Output{array of one hot vector}
		\BlankLine
		
		dictExtract(symptomDict, sentence)\;
		dictExtract(diseaseDict, sentence)\;
		dictExtract(treatmentDict, sentence)\;
		dictExtract(drugDict, sentence)\;
		\BlankLine
		
	}
	
	\caption{\textit{Pseudocode} untuk melakukan ekstraksi fitur kamus kesehatan}	
\end{kode}

\subsubsection{Ekstraksi Frasa Kata Benda}
Dalam mengimplementasikan ektraksi fitur kata benda, \saya~menggunakan \textit{library} NLTK yang mengimplementasikan \textit{chunking}, yang merupakan proses segmentasi dan pelabelan pada \textit{multi-token sequences}. Untuk mengimplementasikannya, \saya~menggunakan informasi POS-Tag yang didapatkan pada implementasi fitur POS Tag, kemudian menentukan \textit{rule} pada proses \textit{chunking} ini. \textit{Rule} yang \saya~gunakan sudah dijelaskan pada Bab 3. Keluaran dari ekstraksi fitur ini yaitu \textit{array of one hot vector} dari masing-masing kata dalam 1 kalimat, dimana apabila suatu kata merupakan bagian dari frasa kata benda akan bernilai [0.0, 1.0], sedangkan yang bukan akan bernilai [1.0, 0.0]. Berikut merupakan \textit{pseudocode} dari implementasi ekstraksi fitur frasa kata benda.

\begin{kode}
	\label{kode:ekstraksinp}
	\SetKwInOut{Input}{Input}
	\SetKwInOut{Output}{Output}
	
	\SetAlgoLined
	\SetKwProg{Fn}{Function}{ is}{end}
	\Fn{npExtract(chunker, sentence, label)}{
		\Input{chunker for a sentence, sentence, label of chunking}
		\Output{array of one hot vector}
		\BlankLine
		
		chunkedSentence = chunker.chunk(sentence)\;
		chunkFeature = []\;
		\ForEach{token in chunkedSentence}{
			\If{token.isLabel(label)}
			{
				chunkFeature.append([0.0, 1.0])\;
			}
			\Else(){
				chunkFeature.append([1.0, 0.0])\;
			}
		}
		\BlankLine
		
		\Return chunkFeature;
		
	}
	\BlankLine
	
	\Fn{main()}{
		corpus = readFile("corpus")\;
		rule = ruleOfNounPhrase\;
		chunker = nltk.RegexpParser(rule)\;
		\BlankLine
		
		corpusChunked = []\;
		\ForEach{sentence in corpus}{
			corpusChunked.append(npExtract(chunker, sentece))\;
		}
		\BlankLine
		
		writeToFile(corpusChunked)
	}
	\BlankLine	
	\caption{\textit{Pseudocode} untuk melakukan ekstraksi fitur kamus kesehatan}	
\end{kode}

\subsubsection{Ekstraksi Frasa Kata Kerja}
Sama seperti pada pengimplementasian ektraksi fitur kata benda, \saya~menggunakan \textit{library} NLTK yang mengimplementasikan \textit{chunking}, yang merupakan proses segmentasi dan pelabelan pada \textit{multi-token sequences}. Untuk mengimplementasikannya, \saya~menggunakan informasi POS-Tag yang didapatkan pada implementasi fitur POS Tag, kemudian menentukan \textit{rule} pada proses \textit{chunking} ini. \textit{Rule} yang \saya~gunakan sudah dijelaskan pada Bab 3. Keluaran dari ekstraksi fitur ini yaitu \textit{array of one hot vector} dari masing-masing kata dalam 1 kalimat, dimana apabila suatu kata merupakan bagian dari frasa kata kerja akan bernilai [0.0, 1.0], sedangkan yang bukan akan bernilai [1.0, 0.0]. Berikut merupakan \textit{pseudocode} dari implementasi ekstraksi fitur frasa kata benda.

\begin{kode}
	\label{kode:ekstraksivp}
	\SetKwInOut{Input}{Input}
	\SetKwInOut{Output}{Output}
	
	\SetAlgoLined
	\SetKwProg{Fn}{Function}{ is}{end}
	\Fn{vpExtract(chunker, sentence, label)}{
		\Input{chunker for a sentence, sentence, label of chunking}
		\Output{array of one hot vector}
		\BlankLine
		
		chunkedSentence = chunker.chunk(sentence)\;
		chunkFeature = []\;
		\ForEach{token in chunkedSentence}{
			\If{token.isLabel(label)}
			{
				chunkFeature.append([0.0, 1.0])\;
			}
			\Else(){
				chunkFeature.append([1.0, 0.0])\;
			}
		}
		\BlankLine
		
		\Return chunkFeature;
		
	}
	\BlankLine
	
	\Fn{main()}{
		corpus = readFile("corpus")\;
		rule = ruleOfVerbPhrase\;
		chunker = nltk.RegexpParser(rule)\;
		\BlankLine
		
		corpusChunked = []\;
		\ForEach{sentence in corpus}{
			corpusChunked.append(npExtract(chunker, sentece))\;
		}
		\BlankLine
		
		writeToFile(corpusChunked)
	}
	\BlankLine	
	\caption{\textit{Pseudocode} untuk melakukan ekstraksi fitur kamus kesehatan}	
\end{kode}

\subsubsection{Ekstraksi Fitur 1 Kata Sebelum}
Ekstraksi fitur ini \saya~lakukan dengan menggunakan hasil dari ekstraksi fitur kata itu sendiri, yaitu mengambil vektor kata dengan indeks saat ini dikurangi satu. Untuk awal kalimat, \saya~memberikan vektor $ \vec{0} $ dimana setiap elemen di dalam \textit{array} merupakan bilangan nol. \ref{kode:ekstraksi1wordbef} merupakan implementasi dari ekstraksi fitur 1 kata sebelum.

\subsubsection{Ekstraksi Fitur 1 Kata Sesudah}
Ekstraksi fitur 1 kata sesudah yang \saya~lakukan ini mirip dengan ekstraksi fitur 1 kata sebelum, perbedaannya pada indeks yang diambil dalam pada saat ekstraksi. Untuk masing-masing kata, \saya~mengambil vektor kata dengan indeks 1 kata setelahnya. Untuk vektor kata di akhir kalimat, \saya~memberikan vektor $ \vec{0} $ dimana setiap elemen di dalam \textit{array} merupakan bilangan nol. \ref{kode:ekstraksi1wordaft} merupakan implementasi dari ekstraksi fitur 1 kata sesudah.

\begin{kode}
	\label{kode:ekstraksi1wordbef}
	\caption{\textit{Pseudocode} untuk melakukan ekstraksi fitur 1 kata sebelum}
	
	\SetKwInOut{Input}{Input}
	\SetKwInOut{Output}{Output}
	
	\SetAlgoLined
	\SetKwProg{Fn}{Function}{ is}{end}
	\Fn{oneWordBeforeExtract(sentenceVector)}{
		\Input{array of vector in a sentence}
		\Output{array of vector}
		\BlankLine
		
		oneWordBeforeFeature = []\;
		oneWordBeforeFeature.append(zeroth)\;
		\For{i in 1...sentenceVector.length}{
			oneWordBeforeFeature.append(sentenceVector[i-1])\;
		}
		\BlankLine
		
		\Return oneWordBeforeFeature;
		
	}

	\BlankLine		
\end{kode}

\begin{kode}
	\label{kode:ekstraksi1wordaft}
	\SetKwInOut{Input}{Input}
	\SetKwInOut{Output}{Output}
	
	\SetAlgoLined
	\SetKwProg{Fn}{Function}{ is}{end}
	\Fn{oneWordBeforeExtract(sentenceVector)}{
		\Input{array of vector in a sentence}
		\Output{array of vector}
		\BlankLine
		
		oneWordAfterFeature = []\;
		\For{i in 0...sentenceVector.length - 1}{
			oneWordAfterFeature.append(sentenceVector[i+1])\;
		}
		oneWordAfterFeature.append(zeroth)\;
		\BlankLine
		
		\Return oneWordAfterFeature;
		
	}
	
	\BlankLine	
	\caption{\textit{Pseudocode} untuk melakukan ekstraksi fitur 1 kata sesudah}	
\end{kode}


\subsubsection{Ekstraksi Fitur 2 Kata Sebelum}
Ekstraksi fitur ini \saya~lakukan dengan menggunakan hasil dari ekstraksi fitur kata itu sendiri, yaitu mengambil vektor kata dengan indeks saat ini dikurangi satu. Untuk dua kata pertama dalam kalimat, \saya~memberikan vektor $ \vec{0} $ dimana setiap elemen di dalam \textit{array} merupakan bilangan nol. Berikut merupakan implementasi dari ekstraksi fitur 2 kata sebelum.

\begin{kode}
	\label{kode:ekstraksi2wordbef}
	\SetKwInOut{Input}{Input}
	\SetKwInOut{Output}{Output}
	
	\SetAlgoLined
	\SetKwProg{Fn}{Function}{ is}{end}
	\Fn{oneWordBeforeExtract(sentenceVector)}{
		\Input{array of vector in a sentence}
		\Output{array of vector}
		\BlankLine
		
		oneWordBeforeFeature = []\;
		oneWordBeforeFeature.append(zeroth)\;
		oneWordBeforeFeature.append(zeroth)\;
		\For{i in 2...sentenceVector.length}{
			oneWordBeforeFeature.append(sentenceVector[i-1])\;
		}
		\BlankLine
		
		\Return oneWordBeforeFeature;
		
	}
	
	\BlankLine	
	\caption{\textit{Pseudocode} untuk melakukan ekstraksi fitur 2 kata sebelum}	
\end{kode}

\subsection{Pengusulan Arsitektur RNNs}
Sesuai dengan yang telah dijelaskan pada Bab 3, \saya~mengusulkan dua arsitektur RNNs yang akan digunakan pada tahap eksperimen. Pada bagian ini \saya~akan menjelaskan implementasi dari masing-masing arsitektur tersebut. Dalam melakukan implementasi RNNs, \saya~menggunakan \textit{library} Keras dalam bahasa Python. Keras sendiri dapat berjalan di atas dua \textit{library deep learning} lain, yaitu Theano dan Tensorflow, namun dalam penelitian ini \saya~menggunakan Theano. \Saya~menggunakan \textit{Sequential model} yang merupakan layer \textit{linear stack} dalam mengembangkan model dan jenis RNNs yang \saya~gunakan dalam penelitian ini adalah LSTM.

\subsubsection{LSTM 1 layer}
LSTM 1 layer yang dimaksud adalah model yang digunakan memiliki satu layer LSTM saja dan semua fitur yang menjadi input program digabung terlebih dahulu menjadi satu buah \textit{array}. Seperti yang telah dijelaskan pada Bab 3, susunan layer yang \saya~gunakan terdiri dari \textit{Masking Layer}, LSTM Layer, dan \textit{Time Distributed Layer} yang masing-masing \textit{timestep} berisi \textit{Dense Layer}. Untuk \textit{Masking Layer}, dimensi yang menjadi parameter tergantung dari \textit{array} yang menjadi masukan, untuk LSTM Layer, dimensi masukan sama dengan dimensi \textit{Masking Layer} dan dimensi keluaran untuk masing-masing \textit{timesteps} adalah panjang input dalam satu \textit{timestep} dibagi 2. Untuk masing-masing Dense Layer, dimensi masukan yang diminta sama dengan dimensi keluaran pada LSTM Layer dan dimensi keluaran sesuai dengan jumlah kelas yang telah didefinisikan.

Masukan yang diminta yaitu \textit{array} yang masing-masing elemennya merupakan \textit{array} dari vektor fitur dan sudah digabung menjadi satu. Keluaran yang diminta merupakan hasil dari pelabelan otomatis dari program ini. Berikut merupakan kode untuk mengimplementasikan model ini.

\begin{kode}
	\label{kode:lstm1}
	\SetKwInOut{Input}{Input}
	\SetKwInOut{Output}{Output}
	
	\SetKwProg{Fn}{Function}{ is}{end}
	\Fn{lstm1(arrTraining, arrTesting)}{
		\Input{training data, testing data}
		\Output{predicted label}
		\BlankLine
		
		shape = arrTraning.shape()\;
		model = Sequential()\;
		model.add(Masking(input\char`_shape:shape))]\;
		model.add(LSTM(output = shape/2))\;
		model.add(TimeDistributed(Dense(output = 9)))\;
		\BlankLine
		
		model.input(arrTraining)\;
		prediction = model.predict(arrTesting)\;
		\BlankLine
		
		\Return prediction;
	}
	
	\caption{\textit{Pseudocode} untuk arsitektur RNNs pertama}	
\end{kode}

\subsubsection{LSTM Layer Bertingkat}
LSTM layer bertingkat yang dimaksud yaitu terdapat dua tingkat, tingkat pertama untuk menerima \textit{input} yang setiap kelompok fitur menjadi input bagi LSTM masing-masing. Misalnya terdapat 3 kelompok fitur, masing-masing kelompok tadi akan menjadi input bagi layer LSTM masing-masing. Tingkat kedua sebagai penggabung hasil dari tingkat pertama.

Layer pada tingkat pertama terdiri dari \textit{Masking Layer} dan sebuah Layer LSTM. Untuk dimensi \textit{input} dan \textit{output} \textit{Masking Layer} secara otomatis mengikuti dimensi dari data masukan. Dimensi \textit{output} dari Layer LSTM yaitu dimensi awal dibagi 2.Pada layer tingkat kedua, layer tersebut terdiri dari \textit{Merge Layer}, \textit{Time Distributed} dengan masing-masing \textit{timestep} merupakan \textit{Dense Layer} dan sebuah Layer LSTM. Keluaran dari \textit{Merge Layer} sesuai dengan total dimensi \textit{output} dari masing-masing LSTM di tingkat 1. Dimensi keluaran dari masing-masing \textit{Dense Layer} yaitu 32 dan sesuai jumlah kelas.
Masukan yang diminta yaitu \textit{array} yang masing-masing elemennya merupakan \textit{array} dari vektor fitur dan sudah digabung menjadi satu. Keluaran yang diminta merupakan hasil dari pelabelan otomatis dari program ini. Berikut merupakan kode untuk mengimplementasikan model ini.

\begin{kode}
	\label{kode:lstm2}
	\SetKwInOut{Input}{Input}
	\SetKwInOut{Output}{Output}
	
	\SetKwProg{Fn}{Function}{ is}{end}
	\Fn{lstm2(groupOfArrTraining, groupOfArrTraining)}{
		\Input{grop of training data, group of testing data}
		\Output{predicted label}
		\BlankLine
		
		modelArr = []\;
		\ForEach{groupFeature in groupOfArrTraining}{
			shape = arrTraning.shape()\;
			model = Sequential()\;
			model.add(Masking(input\char`_shape:shape))]\;
			model.add(LSTM(output = shape/2))\;
			modelArr.append(model)\;
		}
		\BlankLine
		
		mainModel = Sequential()\;
		mainModel.add(Merge(mode='concat', modelArr))]\;
		mainModel.add(TimeDistributed(Dense(output = 32)))\;
		mainModel.add(LSTM(output = 32))\;
		mainModel.add(TimeDistributed(Dense(output = 9)))\;
		\BlankLine
		
		mainModel.input(groupOfArrTraining)\;
		prediction = mainModel.predict(groupOfArrTraining)\;
		\BlankLine
		
		\Return prediction;
	}
	
	\caption{\textit{Pseudocode} untuk arsitektur RNNs kedua}	
\end{kode}

\section{Eksperimen}
Pada tahap ini \saya~melakukan eksperimen model yang dikembangkan pada tahap sebelumnya. Sebelum masuk ke tahap eksperimen, \saya~melakukan beberapa tahap pra-eksperimen seperti melakukan pemecahan data sebagai implementasi \textit{cross-fold validation}. \Saya~memecah data menjadi 10 bagian dan disimpan dalam sebuah \textit{array} untuk masing-masing fitur. Berikut merupakan \textit{pseudocode} untuk melakukan pemecahan data

\begin{kode}
	\label{kode:split}
	\SetKwInOut{Input}{Input}
	\SetKwInOut{Output}{Output}
	
	\SetKwProg{Fn}{Function}{ is}{end}
	\Fn{splitting(featureArr)}{
		\Input{array of feature}
		\Output{splitted array of feature}
		\BlankLine
		
		lenSplit = len(featureArr)/10\;
		arrSplitted = []\;
		\For{i=0; i<10;i++}{
			start = i * lenSplit\;
			end = (i+1) * lenSplit\;
			arrSplitted.append[start:end]\;
		}
		\BlankLine
		
		\Return arrSplitted;
	}
	
	\caption{\textit{Pseudocode} untuk memecah \textit{data} menjadi 10 bagian}	
\end{kode}

Setelah masing-masing fitur dipecah menjadi 10 bagian, \saya~melakukan penggabungan antar fitur sebagai \textit{input} untuk melakukan eksperimen. Seperti yang dijelaskan pada tahap sebelumnya, \saya~menggunakan dua arsitektur RNNs. Hasil dari eksperimen tersebut ditulis dalam sebuah file dengan format JSON yang nantinya akan menjadi \textit{input} pada tahap evaluasi. Berikut merupakan implementasi eksperimen dengan masing-masing arsitektur tersebut.

\begin{kode}
	\label{kode:eksperimen}
	\SetKwInOut{Input}{Input}
	\SetKwInOut{Output}{Output}
	
	\SetKwProg{Fn}{Function}{ is}{end}
	arrAllFeature = []\;
	\ForEach{feature in arrSplitted}{
		arrAllFeature.join(feature)\;
	}
	\BlankLine
	
	\For{i=0; i<10;i++}{
		training = arrSplitted[0:i] + arrSplitted[i+1:10]
		testing = arrSplitted[i]\;
		
		\BlankLine
		result1 = lstm1(training, testing)\;
		result2 = lstm2(training, testing)\;
		
		\BlankLine
		writeToJSON(result1)\;
		writeToJSON(result2)\;
	}
	\caption{\textit{Pseudocode} untuk melakukan eksperimen}	
\end{kode}


\section{Evaluasi}
Dalam melakukan implementasi pada tahap evaluasi, \saya~menghitung nilai \textit{prescission, recall} dan \textit{F-Measures} untuk mengukur tingkat keakuratan model yang dikembangkan pada tahap sebelumnya. \Saya~menggunakan aturan yang telah dijelaskan pada Bab 3. Berikut merupakan implementasi kode untuk melakukan evaluasi.

\begin{kode}
	\label{kode:evaluasi}
	\SetKwInOut{Input}{Input}
	\SetKwInOut{Output}{Output}
	
	\SetKwProg{Fn}{Function}{ is}{end}
	resultTag = load(resultRNN)\;
	originalTag = load(originalTag)\;
	\BlankLine
	
	TP = newHash()\;
	FP = newHash()\;
	FN = newHash()\;
	\For{i = 0; i < len(resultTag); i++}{
		sentenceResult = resultTag[i]\;
		sentenceOriginal = originalTag[i]\;
		\For{j = 0; j < len(sentenceOriginal); i++}{
			wordResult = sentenceResult[j]\;
			wordOri = sentenceOriginal[j]\;
			\uIf{wordOri != O}{
				\uIf{wordResult != O}{
					\uIf{wordOri == wordResult}{
						TP[wordOri] += 1\;
					}
					\Else{
						FN[wordOri] += 1\;
					}
				}
				\Else{
					FN[wordOri] += 1\;
				}
			}
			\Else{
				\uIf{wordResult != O}{
					FP[wordOri] += 1\;
				}
			}
		}
	}
	\BlankLine

	prec = newHash()\;
	rec = newHash()\;
	fMeas = newHash()\;
	\ForEach{label in TP}{
		prec[label] = TP[label] / (TP[label] + FP[label])\;
		rec[label] = TP[label] / (TP[label] + FN[label])\;
		fMeas[label] = 2 * 	(prec[label] * rec[label]) / (prec[label] + rec[label])\;
	}
	\BlankLine
	
	\ForEach{label in prec}{
		print "Precission", label, prec[label]\;
		print "Recall", label, rec[label]\;
		print "F-Measure", label, fmeas[label]\;
	}
	\BlankLine
	\caption{\textit{Pseudocode} untuk melakukan evaluasi}	
	\end{kode}
