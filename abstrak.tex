%!TEX root = skripsi.tex
%
% Halaman Abstrak
%
% @author  Andreas Febrian
% @version 1.00
%

\chapter*{Abstrak}

\vspace*{0.2cm}

\noindent \begin{tabular}{l l p{10cm}}
	Nama&: & \penulis \\
	Program Studi&: & \program \\
	Judul&: & \judul \\
\end{tabular} \\ 

\vspace*{0.5cm}

\noindent
Saat ini, seseorang dapat memanfaatkan forum kesehatan \textit{online} untuk mencari tahu perihal penyakit tanpa perlu tatap muka.  Melalui forum tersebut, seseorang hanya perlu menuliskan keluhan dan pertanyaan pada formulir yang tersedia. Banyak sekali informasi bermanfaat yang dapat diperoleh dari forum tersebut seperti keluhan, obat atau langkah penyembuhan. Melalui penelitian ini, akan diambil informasi \textit{disease, symptom, treatment} dan \textit{drug} secara otomatis yang dinamakan \mer~dengan menggunakan model RNNs. Hal ini karena RNNs merupakan \textit{state of the art} untuk permasalahan \textit{sequence lablelling} seperti permasalahan \mer. Penelitian ini akan mencari fitur-fitur yang kontributif serta arsitektur RNNs yang memberikan akurasi terbaik. Hasil akhir penelitian ini didapatkan bahwa dengan fitur kata itu sendiri, kamus kesehatn, \textit{stop word}, frasa kata, 1 kata sebelum dan 1 kata sesudah akan memberikan hasil yang terbaik, yaitu \textbf{Hasil mana yang harus ditampilkan?}


\vspace*{0.2cm}

\noindent Kata Kunci: \\ 
\noindent \mer, RNNs, \textit{disease}, \textit{symptom}, \textit{treatment}, \textit{drug} \\ 

\newpage