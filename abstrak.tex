%!TEX root = skripsi.tex
%
% Halaman Abstrak
%
% @author  Andreas Febrian
% @version 1.00
%

\chapter*{Abstrak}

\vspace*{0.2cm}

\noindent \begin{tabular}{l l p{10cm}}
	Nama&: & \penulis \\
	Program Studi&: & \program \\
	Judul&: & Semantic Role Labeling untuk Bahasa Indonesia Percakapan dengan Menggunakan Deep Learning \\
\end{tabular} \\ 

\vspace*{0.5cm}

\noindent 

\textit{Semantic Role Labeling} (SRL) telah diteliti cukup dalam, terutama untuk Bahasa Inggris yang formal. Akan tetapi, hanya terdapat sedikit riset untuk bahasa percakapan informal, terutama untuk bahasa yang digunakan dalam sistem \textit{chatbot}. Dalam Bahasa Indonesia, kedua jenis bahasa formal dan percakapan masih sedikit diteliti untuk membangun sistem SRL. Pada riset ini, kami berfokus pada menyelesaikan permasalahan SRL untuk Bahasa Indonesia percakapan. Kontribusi kami adalah memperkenalkan set \textit{semantic roles} baru dan mengajukan \textit{attention mechanism} di atas arsitektur \textit{Long Short-Term Memory Networks}. Tantangan dalam Bahasa Indonesia percakapan adalah \textit{slang} dan singkatan, kalimat pendek, dan struktur yang tidak beraturan. Walaupun riset ini merupakan \textit{pilot task}, kami memperoleh hasil yang menjanjikan dengan nilai F1 82.68\%.


\vspace*{0.2cm}

\noindent Keywords: \\ 
\noindent Semantic Role Labeling, deep learning, bahasa percakapan, RNNs \\ 

\newpage