%!TEX root = skripsi.tex
%
% Halaman Abstrak
%
% @author  Andreas Febrian
% @version 1.00
%

\chapter*{Abstrak}

\vspace*{0.2cm}

\noindent \begin{tabular}{l l p{10cm}}
	Nama&: & \penulis \\
	Program Studi&: & \program \\
	Judul&: & \judul \\
\end{tabular} \\ 

\vspace*{0.5cm}

\noindent
Banyak umat Muslim yang ingin menghafalkan \quran. Namun orang yang menghafalkan \quran~membutuhkan rekan untuk membantu mengevaluasi hafalannya. Untuk membantu proses tersebut, penelitian ini mengembangkan sebuah sistem yang mampu mengevaluasi pembacaan \quran~secara otomatis. Sistem tersebut menggunakan fitur \f{mel frequency cepstral coefficient} (MFCC) dan \f{shifted delta cepstral coefficient} (SDCC), dengan metode klasifikasi \f{support vector machine} (SVM) dan \f{Gaussian mixture model} (GMM). Eksperimen dalam penelitian ini dilakukan terhadap setiap ayat di juz 30 \quran, dan hasilnya menunjukkan bahwa kombinasi yang paling tepat untuk digunakan dalam sistem tersebut adalah fitur SDCC dengan metode klasifikasi GMM. \\

% \quran~adalah kitab suci bagi umat Muslim yang diturunkan dalam Bahasa Arab. Di dalam \quran~terkandung berbagai macam hakikat kehidupan. Banyak umat Muslim yang ingin belajar membaca \quran~serta menghafalkan \quran. Orang yang baru belajar membaca \quran~atau orang yang sedang menghafalkannya membutuhkan rekan yang mengerti cara membaca \quran. Peran rekan tersebut adalah untuk mengevaluasi bacaan orang yang sedang belajar \quran~atau yang sedang menghafalkan \quran. Namun rekan untuk mengevaluasi tidak setiap saat ada. Oleh karena itu, penelitian ini dilakukan dalam rangka membangun sistem pengenalan suara yang mampu mengevaluasi pembacaan \quran~secara otomatis, guna membantu proses evaluasi tersebut. Sistem yang dikembangkan dalam penelitian ini mengekstrak fitur MFCC dan SDCC dari audio. Fitur tersebut diklasifikasi menggunakan metode klasifikasi SVM dan GMM. Eksperimen dilakukan terhadap ayat-ayat di juz 30 \quran. Hasil eksperimen menunjukkan bahwa kombinasi yang paling tepat untuk digunakan dalam sistem ini adalah penggunaan fitur SDCC dan metode klasifikasi GMM. \\

% \\ \quran~adalah kitab suci bagi umat islam. Di dalamnya terkandung berbagai macam hakikat kehidupan. Oleh karena itu, banyak umat islam yang ingin menghafalkan \quran. \quran~diturunkan dalam Bahasa Arab, dan dalam Bahasa Arab pula dihafalkan. Cara menghafalkan \quran~ada berbagai macam. Salah satunya adalah dengan meminta bantuan orang kedua untuk menyimak hafalannya. Ketika ada bacaan yang salah dari orang pertama, maka orang kedua mengingatkan atau memberi tanda kepada orang pertama. Namun hal itu tidak dapat dilakukan jika tidak ada orang kedua. Cara lain adalah dengan membaca beberapa ayat, lalu mencocokkan yang dibaca dengan yang ada di \quran. Hal itu selain menghambat proses menghafal karena harus sering melihat ke teks \quran, juga memberi peluang orang yang menghafal melihat bagian selanjutnya yang seharusnya tidak boleh dia lihat. Kemudian untuk membuat proses menghafal lebih praktis, maka dikembangkan sistem komputer yang mampu menggantikan peran sebagai orang kedua, yaitu mencocokkan bacaan yang dibaca oleh penghafal dengan teks \quran~yang benar. Sistem ini menggunakan Pengenalan Suara Otomatis (PSO), atau dalam Bahasa Inggris dikenal sebagai {\em Automatic Speech Recognition} (ASR).



\vspace*{0.2cm}

\noindent Kata Kunci: \\ 
\noindent \quran, evaluasi, MFCC, SDCC, SVM, GMM \\ 

\newpage