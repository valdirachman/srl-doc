%!TEX root = skripsi.tex
%
% Halaman Abstrak
%
% @author  Andreas Febrian
% @version 1.00
%

\chapter*{Abstrak}

\vspace*{0.2cm}

\noindent \begin{tabular}{l l p{10cm}}
	Nama&: & \penulis \\
	Program Studi&: & \program \\
	Judul&: & \judul \\
\end{tabular} \\ 

\vspace*{0.5cm}

\noindent
Saat ini, seseorang dapat memanfaatkan forum kesehatan \textit{online} untuk mencari tahu perihal penyakit tanpa perlu tatap muka.  Melalui forum tersebut, seseorang hanya perlu menuliskan keluhan dan pertanyaan pada formulir yang tersedia. Banyak sekali informasi bermanfaat yang dapat diperoleh dari forum tersebut seperti keluhan, obat atau langkah penyembuhan. Penelitian kali ini mencoba untuk melakukan ekstraksi entitas \textit{disease, symptom, treatment} dan \textit{drug} secara otomatis. \Saya memandang permasalahan ini sebagai permasalahan \textit{sequence labelling}. Kami mengusulkan penggunaan teknik \textit{Deep Learning} dengan menggunakan \textit{Recurrent Neural Networks} (RNNs), karena RNNs merupakan \textit{state-of-the-art} untuk permasalahan \textit{sequence lablelling} seperti permasalahan pada penelitian ini. \Saya~mengusulkan fitur kata itu sendiri, \textit{stop word}, POS-Tag, frasa kata (nomina dan verba), kata sebelum dan kata sesudah. Selain itu \saya~juga mengusulkan dua arsitektur RNNs, yaitu LSTMs 1 layer dan LSTMs 2 layer. Hasil eksperimen menunjukkan bawah model yang diusulkan mampu memberikan hasil yang cukup baik. Berdasarkan eksperimen dengan kombinasi fitur kata itu sendiri, \textit{stop word},frasa kata (nomina dan verba), 1 kata sebelum dan 1 kata sesudah, arsitektur RNNs 1 layer, mampu mencapai rata-rata \textit{f-measure} .... (\textit{disease}, ...).
Ada yang perlu diperhatikan bahwa perfoorma model yang kami usulkan lebih baik dari baseline.

H Penelitian ini akan mencari fitur-fitur yang kontributif serta arsitektur RNNs yang memberikan akurasi terbaik. Hasil akhir penelitian ini didapatkan bahwa dengan fitur kata itu sendiri, kamus kesehatn, \textit{stop word}, frasa kata, 1 kata sebelum dan 1 kata sesudah akan memberikan hasil yang terbaik, yaitu \textbf{Hasil mana yang harus ditampilkan?}


\vspace*{0.2cm}

\noindent Kata Kunci: \\ 
\noindent \mer, RNNs, \textit{disease}, \textit{symptom}, \textit{treatment}, \textit{drug} \\ 

\newpage