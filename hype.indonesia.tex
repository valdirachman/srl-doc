%!TEX root = skripsi.tex
%
% Hyphenation untuk Indonesia 
%
% @author  Andreas Febrian
% @version 1.00
% 
% Tambahkan cara pemenggalan kata-kata yang salah dipenggal secara otomatis 
% oleh LaTeX. Jika kata tersebut dapat dipenggal dengan benar, maka tidak 
% perlu ditambahkan dalam berkas ini. Tanda pemenggalan kata menggunakan 
% tanda '-'; contoh:
% menarik
%   --> pemenggalan: me-na-rik
%

\hyphenation{
    % alphabhet A
    a-na-li-sa a-tur 
    a-pli-ka-si
    al-fa-bet
    al-fa-nu-me-rik
    % alphabhet B
    ba-ngun-an 
    be-be-ra-pa 
    ber-ge-rak
    ber-ke-lan-jut-an 
    ber-pe-nga-ruh 
    % alphabhet C
    ca-ri
    % alphabhet D
    di-ban-ding-kan
    di-de-fi-ni-si-kan
    di-ha-rap-kan
    di-ka-te-go-ri-kan
    di-mi-li-ki-nya
    di-se-rah-kan
    di-sim-pan di-pim-pin de-ngan da-e-rah di-ba-ngun da-pat di-nya-ta-kan 
    di-sim-bol-kan di-pi-lih di-li-hat de-fi-ni-si
    di-se-su-ai-kan
    % alphabhet E
    e-le-men
    e-ner-gi eks-klu-sif
    % alphabhet F
    fa-si-li-tas
    % alphabhet G
    ga-bung-an ge-rak
    % alphabhet H
    ha-lang-an
    he-te-ro-gen
    % alphabhet I
    i-ngin
    % alphabhet J
    % alphabhet K
    ke-hi-lang-an
    ku-ning 
    kua-li-tas ka-me-ra ke-mung-kin-an ke-se-pa-ham-an
    ke-te-pat-an
    kon-fi-gu-ra-si
    % alphabhet L
    ling-kung-an
    % alphabhet M
    me-min-ta
    me-mo-del-kan
    me-mo-ri
    men-de-fi-ni-si-kan
    me-neng-ah
    meng-a-tas-i me-mung-kin-kan me-nge-na-i me-ngi-rim-kan 
    meng-u-bah meng-a-dap-ta-si me-nya-ta-kan mo-di-fi-ka-si
    meng-a-tur
    meng-au-to-ma-si
    meng-a-ko-mo-da-si
    me-ngo-rek-si
    % alphabhet N
    nya-ta non-eks-klu-sif
    nu-me-rik
    % alphabhet O
    % alphabhet P
    pa-ra-lel
    peng-ala-mat-an
    pen-ting
    penga-da-an
		pe-nye-rap-an 
		pe-ngon-trol
    pe-mo-del-an
    pe-ran  pe-ran-an-nya
    pe-rin-tah
    pem-ba-ngun-an pre-si-den pe-me-rin-tah prio-ri-tas peng-am-bil-an 
    peng-ga-bung-an pe-nga-was-an pe-ngem-bang-an 
    pe-nga-ruh pa-ra-lel-is-me per-hi-tung-an per-ma-sa-lah-an 
    pen-ca-ri-an peng-struk-tur-an
    pe-ner-bang-an
    pro-se-sor
    % alphabhet Q
    % alphabhet R
    ran-cang-an
    % alphabhet S
    se-dang-kan
    se-ring
    si-mu-la-si sa-ngat    
    % alphabhet T
    te-ngah
    ter-da-pat
    % alphabhet U
    u-sa-ha
    % alphabhet V
    % alphabhet W
    % alphabhet X
    % alphabhet Y
    % alphabhet Z
    % special
    MATLAB
    basmalah
    sam-ping
    hardware
    software
    voiced
    overlap
    frame
    bit
    rate
    support
    vector
    machine
    Gaussian
    mixture
    model
    dynamic
    pro-gram-ming
    auto-ma-tically
    mem-ba-ca
    teks-tual
    eks-trak-si
    di-je-las-kan
    be-nar
    sa-lah
    teks-tual
    dengan
    me-re-pre-sen-ta-si
    me-re-pre-sen-ta-si-kan
}