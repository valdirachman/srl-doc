%!TEX root = skripsi.tex
%-----------------------------------------------------------------------------%
\chapter{\babEnam}
%-----------------------------------------------------------------------------%

%-----------------------------------------------------------------------------%
\section{Kesimpulan}
%-----------------------------------------------------------------------------%
Setelah mengimplementasikan rancangan arsitektur sistem, menjalankan eksperimen, serta menganalisis hasil eksperimen, diperoleh kesimpulan sebagai berikut.

\begin{enumerate}
  \item Penggunaan fitur SDCC berpeluang lebih baik daripada fitur MFCC. Hal ini dapat diamati dari Gambar \ref{fig:pieclass0}, Gambar \ref{fig:pieclass1}, dan Gambar \ref{fig:pieclass2}. Ketiga gambar tersebut menunjukkan hasil yang konsisten bahwa fitur SDCC lebih mendominasi dalam memberikan hasil yang lebih akurat daripada fitur MFCC.

  \item Penggunaan metode klasifikasi GMM berpeluang lebih baik daripada metode klasifikasi SVM maupun metode gabungan SVM dengan GMM. Hal ini dapat diamati dari \ref{fig:piefeat0} dan Gambar \ref{fig:piefeat1}. Kedua gambar tersebut menunjukkan bahwa metode klasifikasi GMM lebih mendominasi dalam memberikan hasil yang paling akurat dibandingkan metode klasifikasi lainnya.

  \item Kombinasi pengambilan fitur SDCC dengan metode klasifikasi GMM adalah kombinasi yang memberikan hasil paling akurat secara rata-rata dalam penelitian ini. Hal ini dapat diamati dari nilai rata-rata akurasi pada Tabel \ref{table:mfccsvm}, Tabel \ref{table:mfccgmm}, Tabel \ref{table:mfccgabungan}, Tabel \ref{table:sdccsvm}, Tabel \ref{table:sdccgmm}, dan Tabel \ref{table:sdccgabungan}. Kombinasi fitur SDCC dengan metode klasifikasi GMM menghasilkan nilai rata-rata akurasi tertinggi, yaitu sebesar 83,4\%.

  \item Penggabungan dua metode klasifikasi yang sama-sama memiliki kinerja yang baik tidak menjamin akan menghasilkan metode klasifikasi baru yang lebih akurat secara rata-rata. Nilai rata-rata akurasi dari metode klasifikasi GMM turun setelah digabung dengan metode klasifikasi SVM, baik pada pengambilan fitur MFCC maupun pada pengambilan fitur SDCC.

  \item Akurasi dari masing-masing ayat berbeda-beda. Ada beberapa ayat yang dapat diklasifikasikan dengan akurasi tinggi, dan ada juga beberapa ayat yang hanya diklasifikasikan dengan akurasi rendah.
\end{enumerate}

%-----------------------------------------------------------------------------%
\section{Saran}
%-----------------------------------------------------------------------------%
Setelah melakukan eksperimen dan menganalisis hasilnya, ada beberapa saran untuk penelitian selanjutnya, antara lain sebagai berikut.

\begin{enumerate}
  \item Fitur MFCC dan fitur SDCC memungkinkan untuk dikombinasikan. Kombinasi tersebut layak untuk dicoba dalam penelitian selanjutnya karena ada peluang akurasi dari sistem akan meningkat.

  \item Sistem dalam penelitian ini dapat dikembangkan lebih lanjut dengan cara mensegmentasi ayat-ayat yang panjang, seperti yang ada dalam Surat Al-Baqarah. Dalam surat tersebut terdapat ayat sepanjang satu halaman \quran. Dengan cara disegmentasi, ayat-ayat yang panjang dapat diperlakukan seperti ayat-ayat yang pendek di juz 30. Sehingga sistem yang dikembangkan dalam penelitian ini dapat digunakan untuk seluruh ayat dalam \quran.

  \item Banyaknya data sampel dapat ditingkatkan, tidak hanya 40 sampel qari. Data sampel yang semakin banyak diharapakan akan menghasilkan sistem yang semakin akurat dan presisi.

  \item Masih ada metode-metode klasifikasi lain yang layak untuk dicoba dalam penelitian selanjutnya, seperti \f{deep learning}, \f{i-vector}, dan \f{Gaussian process}. Ada kemungkinan metode klasifikasi lain akan menghasilkan sistem yang lebih akurat dan presisi.

  \item Metode klasifikasi yang dilakukan dalam penelitian ini masih menggunakan parameter yang konstan. Ada potensi untuk meningkatkan akurasi sistem dengan memilih parameter yang lebih tepat pada setiap metode klasifikasi.

  \item Penelitian ini dapat dikembangkan untuk mencari tahu qari mana yang pelafalannya paling sering diidentifikasi dengan benar.
\end{enumerate}