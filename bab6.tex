%!TEX root = skripsi.tex
%-----------------------------------------------------------------------------%
\chapter{\babEnam}
%-----------------------------------------------------------------------------%

%-----------------------------------------------------------------------------%
\section{Kesimpulan}
%-----------------------------------------------------------------------------%
Penelitian ini bertujuan untuk membangun sistem yang mampu melakukan ekstraksi entitas kesehatan dari forum \textit{online} secara otomatis. data yang digunakan bersumber dari \textit{oost} pada forum kesehatan \textit{online} yang telah didapatkan oleh \cite{skripsiKakRadit} dan \saya. Fitur yang digunakan pada penelitian ini yaitu fitur kata itu sendiri, kamus kesehatan, \textit{stop word}, POS-Tag, frasa kata,  1 kata sebelum dan 1 kata sesudah. Penelitian ini juga menggunakan dua arsitektur RNNs yang diusulkan, yaitu LSTMs dengan 1 layer dan LSTMs dengan 2 layer.

Setelah dilakukan penelitian, secara garis besar didapatkan bahwa fitur kata itu sendiri, kamus kesehatan, \textit{stop word}, frasa Kata,  1 kata sebelum dan 1 kata sesudah memberikan hasil yang terbaik, yaitu dengan rata-rata \textit{f-measure} $ 63.06\% $ (\textit{disease} $ 68.17\% $, \textit{symptom} $ 61.42\% $, \textit{treatment} $ 68.17\% $ dan \textit{drug} $ 68.17\% $).

Dua arsitektur yang diusulkan memiliki kelebihan masing-masing. Untuk arsitektur LSTMs dengan 1 layer, \textit{f-measure} sama dengan percobaan untuk mendapatkan fitur terbaik, karena eksperimen tersebut menggunakn arsitektur LSTMs yang sama. Sedangkan untuk arsitektur kedua (LSTMs 2 layer), rata-rata \textit{f-measure} yang didapatkan adalah $ 62.14\% $. LSTMs pertama memiliki nilai \textit{f-measure} pada entitas \textit{disease} dan \textit{drug} yang lebih bagus, yaitu masing-masing $ 68.17\% $ dan $ 69.82\% $. Sedangkan LSTMs kedua memiliki nilai \textit{f-measure} pada entitas \textit{symptom} $ 62.13\% $ dan \textit{treatment} $ 56.51\% $. Namun, apabila dilihat dari waktu komputasi, LSTMs pertama lebih baik dibandingkan LSTMs kedua.

%-----------------------------------------------------------------------------%
\section{Saran}
%-----------------------------------------------------------------------------%
Setelah melakukan eksperimen dan menganalisis hasilnya, ada beberapa saran untuk penelitian selanjutnya, antara lain sebagai berikut.

\begin{enumerate}
  \item Penelitian ini hanya menggunakan 305 \textit{post} forum kesehatan \textit{online} sehingga perlu penambahan data \textit{training} dan \textit{testing} mengingat \textit{deep learning} membutuhkan data yang besar dalam melakukan \textit{training} untuk mendapatkan model yang baik.

  \item Perlu dibuat korpus dengan jumlah masing-masing entitas yang seimbang, sehingga hasil yang diberikan tidak bias.
  
  \item Sebaiknya, pelabelan dokumen secara manual melibatkan pihak yang ahli di bidangnya (dalam hal ini dokter, perawat, apoteker, atau mahasiswa di bidang kesehatan) supaya label yang diberikan lebih tepat.
  
  \item Sama seperti pada penelitian \cite{skripsiKakRadit}, sebaiknya dibuat model POS-Tagger yang khusus di bidang kesehatan, sehingga pemberian tag pada dokumen kesehatan lebih tepat.

\end{enumerate}