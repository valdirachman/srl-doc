%!TEX root = skripsi.tex
%-----------------------------------------------------------------------------%
\chapter{\babEnam}
%-----------------------------------------------------------------------------%

%-----------------------------------------------------------------------------%
\section{Kesimpulan}
%-----------------------------------------------------------------------------%
Peneitian ini bertujuan untuk membangun sistem yang mampu melakukan ekstraksi entitas kesehatan dari forum \textit{online} secara otomatis. Fitur yang digunakan pada penelitian ini yaitu fitur Kata itu Sendiri, Kamus Kesehatan, \textit{Stop Word}, POS-Tag, Frasa kata,  1 kata sebelum dan 1 kata sesudah. Penelitian ini juga menggunakan dua arsitektur RNNs yang diusulkan, yaitu RNNs dengan 1 layer dan RNNs dengan 2 layer.

Setelah dilakukan penelitian, secara garis besar didapatkan bahwa fitur Kata itu Sendiri, Kamus Kesehatan, \textit{Stop Word}, Frasa Kata,  1 Kata Sebelum dan 1 Kata Sesudah memberikan hasil yang terbaik, yaitu \textbf{angka yang manakah yang perlu ditampilkan? Secara overall atau diperici bagian terbaik untuk entitas \textit{disease}, \textit{symptom}, \textit{treatment}, \textit{drug}, atau \textit{overall}}.

Selain itu, arsitektur yang \saya~usaulkan dalam penelitian ini, yaitu arsitektur RNNs dengan 2 layer memberikan hasil yang terbaik pada penelitian ini. Selain itu, hasil yang diberikan juga lebih baik dibandingkan dengan hasil penelitian \cite{skripsiKakRadit}. \textbf{angka yang manakah yang perlu ditampilkan? Secara overall atau diperici bagian terbaik untuk entitas \textit{disease}, \textit{symptom}, \textit{treatment}, \textit{drug}, atau \textit{overall}}.

%-----------------------------------------------------------------------------%
\section{Saran}
%-----------------------------------------------------------------------------%
Setelah melakukan eksperimen dan menganalisis hasilnya, ada beberapa saran untuk penelitian selanjutnya, antara lain sebagai berikut.

\begin{enumerate}
  \item Penelitian ini hanya menggunakan 305 \textit{post} forum kesehatan \textit{online} sehingga perlu penambahan data \textit{training} dan \textit{testing} mengingat \textit{deep learning} membutuhkan data yang besar dalam melakukan \textit{training}.

  \item Perlu dibuat korpus dengan jumlah masing-masing entitas yang seimbang, sehingga hasil yang diberikan tidak bias.

\end{enumerate}