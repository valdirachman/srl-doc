%!TEX root = skripsi.tex
%-----------------------------------------------------------------------------%
\chapter{\babEnam}
%-----------------------------------------------------------------------------%

%-----------------------------------------------------------------------------%
\section{Kesimpulan}
%-----------------------------------------------------------------------------%
Terkait dengan rumusan masalah pertama, setelah dilakukan penelitian secara garis besar didapatkan kesimpulan bahwa model RNNs yang dihasilkan mampu memberikan performa yang lebih baik dibandingkan dengan model CRF (\textit{baseline}) pada penelitian \cite{skripsiKakRadit}. Dari penggunaan fitur kata itu sendiri saja, model RNNs sudah memberikan performa yang lebih baik dengan nilai \textit{F-Measure} dan \textit{recall} yang lebih tinggi.

Terkait dengan rumusan masalah kedua, didapatkan kesimpulan lain bahwa fitur kata itu sendiri, kamus kesehatan, \textit{stop word}, frasa Kata,  1 kata sebelum dan 1 kata sesudah memberikan hasil yang terbaik, yaitu dengan rata-rata \textit{f-measure} $ 63.06\% $ (\textit{disease} $ 68.17\% $, \textit{symptom} $ 61.42\% $, \textit{treatment} $ 68.17\% $ dan \textit{drug} $ 68.17\% $).

Dua arsitektur yang diusulkan memiliki kelebihan masing-masing. Untuk arsitektur LSTMs dengan 1 layer, \textit{f-measure} sama dengan percobaan untuk mendapatkan fitur terbaik, karena eksperimen tersebut menggunakn arsitektur LSTMs yang sama. Sedangkan untuk arsitektur kedua (LSTMs 2 layer), rata-rata \textit{f-measure} yang didapatkan adalah $ 62.14\% $. LSTMs pertama memiliki nilai \textit{f-measure} pada entitas \textit{disease} dan \textit{drug} yang lebih bagus, yaitu masing-masing $ 68.17\% $ dan $ 69.82\% $. Sedangkan LSTMs kedua memiliki nilai \textit{f-measure} pada entitas \textit{symptom} $ 62.13\% $ dan \textit{treatment} $ 56.51\% $. Namun, apabila dilihat dari waktu komputasi, LSTMs pertama lebih baik dibandingkan LSTMs kedua.

LSTMs pertama tidak bisa dikatakan lebih baik dibandingkan LSTMs kedua dan begitu pula sebaliknya, karena hasil yang diperoleh mengatakan bahwa masing-masing arsitektur memiliki hasil yang lebih baik di beberapa macam entitas. Namun, arsitektur ini mampu memberikan hasil yang lebih baik dari hasil penelitian \cite{skripsiKakRadit}. Hal ini akan menarik apabila \textit{resource} semakin diperbesar, apakah tetap LSTMs 1 layer lebih baik, karena LSTMs 2 layer memiliki parameter lebih banyak, sehingga mampu menyimpan informasi yang lebih besar.

%-----------------------------------------------------------------------------%
\section{Saran}
%-----------------------------------------------------------------------------%
Setelah melakukan eksperimen dan menganalisis hasilnya, ada beberapa saran untuk penelitian selanjutnya, antara lain sebagai berikut.

\begin{enumerate}
  \item Penelitian ini hanya menggunakan 309 \textit{post} forum kesehatan \textit{online} sehingga perlu penambahan data \textit{training} dan \textit{testing} mengingat \textit{deep learning} membutuhkan data yang besar dalam melakukan \textit{training} untuk mendapatkan model yang baik.
  
  \item Terdapat beberapa parameter bebas seperti dalam pembuatan model \textit{word embedding} yaitu panjang \textit{window} dan \textit{vector}. Hal ini bisa menjadi bahan penelitian lanjutan untuk mendapatkan panjang \textit{window} dan \textit{vector} yang tepat supaya model mampu memberikan akurasi yang lebih baik.
  
  \item Dalam menentukan label entitas, \saya~tidak mempertimbangkan konteks kalimat yang berada di sekitarnya. Padahal kalimat di sekitarnya akan memberikan informasi lebih terkait hubungan antar entitas. Misalnya pada kalimat pertama dokter menjelaskan penyakit yang dialami. Pada kalimat selanjutnya dokter tersebut menjelaskan cara penyembuhan dari penyakit tersebut. Oleh karena itu, \saya~menyarankan untuk mempertimbangkan fitur konteks kalimat pada penelitian selanjutnya.
  

  \item Perlu dibuat korpus dengan jumlah masing-masing entitas yang seimbang, sehingga hasil yang diberikan tidak bias.
  
  \item Sebaiknya, pelabelan dokumen secara manual melibatkan pihak yang ahli di bidangnya (dalam hal ini dokter, perawat, apoteker, atau mahasiswa di bidang kesehatan) supaya label yang diberikan lebih tepat.
  
  \item Sama seperti pada penelitian \cite{skripsiKakRadit}, sebaiknya dibuat model POS-Tagger yang khusus di bidang kesehatan, sehingga pemberian tag pada dokumen kesehatan lebih tepat.

\end{enumerate}