%!TEX root = skripsi.tex
\section{{KAMUS SYMPTOM}
\begin{longtable}{|p{.25\textwidth}|p{.25\textwidth}|p{.25\textwidth}|p{.25\textwidth}|}

	\caption{Daftar Kata dalam Kamus \textit{Symptom}}\label{lampiran:treatment}\\
	% header and footer information
	\hline
	Pemeriksaan indra penciuman & Inspeksi lebar celah palpebra & Inspeksi pupil & Reaksi pupil terhadap cahaya \\ \hline
	Reaksi pupil terhadap obyek dekat & Penilaian gerakan bola mata & Penilaian diplopia & Penilaian nistagmus \\ \hline
	Refleks kornea & Pemeriksaan fundus kopi & Penilaian kesimetrisan wajah & Penilaian kekuatan otot temporal dan masseter \\ \hline
	Penilaian sensasi wajah & Penilaian pergerakan wajah & Penilaian indra pengecapan & Penilaian indra pendengaran \\ \hline
	Penilaian kemampuan menelan & Inspeksi palatum & Pemeriksaan refleks Gag & Penilaian otot sternomastoid dan trapezius \\ \hline
	inspeksi saat istirahat 4A & inspeksi dan penilaian sistem motorik & Inspeksi: postur,habitus, gerakan involunter & Penilaian tonus otot \\ \hline
	Penilaian kekuatan otot & Inspeksi cara berjalan & Shallow knee bend & Tes Romberg \\ \hline
	Tes Romberg dipertajam & Tes telunjuk hidung & Tes tumit lutut & Tes untuk disdiadokinesis \\ \hline
	Penilaian sensasi nyeri & Penilaian sensasi suhu & Penilaian sensasi raba halus & Penilaian rasa posisi \\ \hline
	proprioseptif & Penilaian sensasi diskriminatif & stereognosis & Penilaian tingkat kesadaran dengan skala koma \\ \hline
	Penilaian orientasi & Penilaian kemampuan berbicara dan berbahasa & Penilaian apraksia & Penilaian agnosia \\ \hline
	Penilaian kemampuan belajar baru & Penilaian daya ingat & Penilaian konsentrasi & Refleks tendon \\ \hline
	Refleks abdominal & Refleks kremaster & Refleks anal & Tanda Hoffmann Tromner \\ \hline
	Respon plantar & Snout reflex & Refleks menghisap & rooting reflex menggenggam palmar \\ \hline
	grasp reflex glabella palmomental & Refleks menggenggam palmar & Refleks glabela & Refleks palmomental \\ \hline
	Inspeksi tulang belakang saat istirahat & Inspeksi tulang belakang saat bergerak & Perkusi tulang belakang & Palpasi tulang belakang \\ \hline
	Mendeteksi nyeri diakibatkan tekanan vertikal & Penilaian fleksilumbal & Deteksi kaku kuduk & Penilaian fontanel \\ \hline
	Tanda Patrick dan kontra-Patrick & Tanda Chvostek & Tanda Lasegue & Interpretasi X-Ray tengkorak \\ \hline
	Interpretasi X-Ray tulang belakang & CT-Scan otak dan interpretasi & EEG dan interpretasi & EMG, EMNG dan interpretasi \\ \hline
	Electronystagmograp(ENG) & hyMRI & PET & SPECT \\ \hline
	Angiography & Duplex-scan pembuluh darah & Punksi lumbal & Therapeutic spinaltap \\ \hline
	Penilaian status mental & Penilaian kesadaran & Penilaian persepsi orientasi intelegensi secara klinis & Penilaian orientasi \\ \hline
	Penilaian intelegensi secara klinis & Penilaian bentuk dan isi pikir & Penilaian mood danafek & Penilaian motorik \\ \hline
	Penilaian pengendalian impuls & Penilaian kemampuan menilai realitas & Penilaian kemampuan tilikan & Penilaian kemampuan fungsional \\ \hline
	general assessment of functioning & Tes kepribadian & Electroconvulsion therapy & ECT \\ \hline
	Psikoterapi suportif & konselling & Psikoterapi modifikasi perilaku & Cognitive Behavior Therapy \\ \hline
	CBT & Psikoterapi psikoanalitik & Hipnoterapi dan terapi relaksasi & Group Therapy \\ \hline
	Family Therapy & Donders confrontation test & Amsler panes & Inspeksi kelopak mata \\ \hline
	Inspeksi kelopak mata dengan eversi kelopak atas & Inspeksi bulu mata & Inspeksi konjungtiva, termasuk forniks & Inspeksi sklera \\ \hline
	Inspeksi orifisi umduktus lakrimalis & Palpasi limfonoduspre-aurikular & Penilaian posisi dengan corneal reflex images & Penilaian posisi dengan cover uncover test \\ \hline
	Pemeriksaan gerakan bola mata & Penilaian penglihatan binokular & Inspeksi pupil & Penilaian pupil dengan reaksi langsung terhadap cahaya dan konvergensi \\ \hline
	Inspeksi media refraksi dengan transilluminasi & Inspeksi kornea & Inspeksi kornea dengan fluoresensi & Tes sensivitas kornea \\ \hline
	Inspeksi bilik mata depan & Inspeksi iris & Inspeksi lensa & Pemeriksaan dengan slit-lamp \\ \hline
	Fundoscopy untuk melihat fundus reflex 4A & Fundoscopy untuk melihat pembuluh darah, papil, makula & Tekanan intraokular, estimasi dengan palpasi & Tekanan intraokular, pengukuran dengan indentasi tonometer \\ \hline
	SchiÃ?utz & Tekanan intraokular, pengukuran dengan aplanasi tonometer & Penentuan refraksi setelah sikloplegia & skiascopy \\ \hline
	Pemeriksaan lensa kontak fundus & gonioscopy & Pengukuran produksi air mata & Pengukuran eksoftalmos \\ \hline
	Pembilasan melalui saluran lakrimalis & Pemeriksaan orthoptic & Perimetri & Pemeriksaan lensa kontak dengan komplikasi \\ \hline
	Tes penglihatan warna & Elektroretinografi & Electrooculography & Visual evoked potentials \\ \hline
	Fluorescein angiography & Echographic examination & ultrasonography & USG \\ \hline
	Inspeksi aurikula & Inspeksi posisi telinga & Inspeksi mastoid & Pemeriksaan meatus auditorius externus \\ \hline
	Pemeriksaan membran timpani & Menggunakan cermin kepala & Menggunakan lampu kepala & Tes pendengaran \\ \hline
	pemeriksaan garpu tala & Tes pendengaran & tes berbisik & Intepretasi hasil Audiometri \\ \hline
	Pemeriksaan pendengaran pada anak-anak & Otoscopy pneumatic & Pemeriksaan vestibular &  \\ \hline
	Tes Ewing & Inspeksi bentuk hidung dan lubang hidung & Penilaian obstruksi hidung & Uji penciuman \\ \hline
	Rinoskopi anterior & Transluminasi sinusfrontalis & maksila & Nasofaringoskopi \\ \hline
	USG sinus & Radiologi sinus & Interpretasi radiologi sinus & Penilaian pengecapan \\ \hline
	Pemberian obat tetes mata & Aplikasi salep mata & Flood ocular tissue & Eversi kelopak atas dengan kapas lidi \\ \hline
	To apply eyes dressing & Melepaskan lensa kontak dengan komplikasi & Melepaskan protesa mata & Mencabut bulu mata \\ \hline
	Membersihkan benda asing & Terapi laser & Operasi katarak & Squint \\ \hline
	surgery & Vitrectomi & Operasi glaukoma dengan trabekulotomi & Transplantasi kornea \\ \hline
	Cryocoagulation & cyclocryocoagulation & Bedah kelopak mata & Operasi detached retina \\ \hline
	Manuver Politzer & Manuver Valsalva & Pembersihan meatus auditorius eksternus dengan usapan & Pengambilan serumen menggunakan kait \\ \hline
	Pengambilan serumen menggunakan kait kuret & Pengambilan benda asing di telinga & Parasentesis & Insersi grommettube \\ \hline
	Menyesuaikan alat bantu dengar & Menghentikan perdarahan hidung & Pengambilan benda asing dari hidung & Bilas sinus \\ \hline
	sinus lavage & pungsi sinus & Antroskopi & Trakeostomi \\ \hline
	Krikotiroidektomi & Inspeksi leher & Palpasi kelenjar ludah & Palpasi nodus limfatikus brakialis \\ \hline
	Palpasi kelenjar tiroid & Rhinoskopi posterior & Laringoskopi indirek & Laringoskopi direk \\ \hline
	Usap tenggorokan & throat swab & Oesophagoscopy & Penilaian respirasi \\ \hline
	Inspeksi dada & Palpasi dada & Perkusi dada & Auskultasi dada \\ \hline
	Gram dan Ziehl Nielsen & Pengambilan cairan pleura & pleural tap & Uji fungsi paru \\ \hline
	spirometri dasar & Tes provokasi bronkial & Interpretasi Rontgen & foto toraks \\ \hline
	Ventilation Perfusion Lung Scanning & Bronkoskopi & FNAB superfisial & Trans thoracal needle aspiration \\ \hline
	TINA & Dekompresi jarum & Pemasangan WSD & Ventilasi tekanan positif pada bayi baru lahir \\ \hline
	Perawatan WSD & Pungsi pleura & Terapi inhalasi & Terapi nebulisasi \\ \hline
	Terapi oksigen & Edukasi berhenti merokok & Inspeksi dada & Palpasi denyut apeks jantung \\ \hline
	Palpasi arteri karotis & Perkusi ukuran jantung & Auskultasi jantung & Pengukuran tekanan darah \\ \hline
	Pengukuran tekanan vena jugularis & JVP & Palpasi denyut arteri ekstremitas & Penilaian denyut kapiler \\ \hline
	Penilaian pengisian ulang kapiler & capillary refill & Deteksi bruits & Tes Trendelenburg \\ \hline
	Tes Perthes & Test Homan & HomanâA ? Zs sign ´ & Uji postur untuk insufisiensi arteri \\ \hline
	Tes hiperemia reaktif untuk insufisiensi arteri & Test ankle-brachial index & ABI & Exercise ECG Testing \\ \hline
	Elektrokardiografi & Ekokardiografi & Fonokardiografi & USG Doppler \\ \hline
	Pijat jantung luar & Resusitasi cairan & Inspeksi bibir & Inspeksi kavitasoral \\ \hline
	Inspeksi tonsil & Penilaian pergerakan otot-otot hipoglosus & Inspeksi abdomen & Inspeksi lipat paha pada saat tekanan abdomen meningkat \\ \hline
	Palpasi dinding perut & Palpasi kolon & Palpasi hepar & Palpasi lien \\ \hline
	Palpasi aorta & Palpasi rigiditas dinding perut & Palpasi hernia & Pemeriksaan nyeri tekan \\ \hline
	Pemeriksaan nyeri lepas & Pemeriksaan psoassign & Pemeriksaan obturator sign & Pemeriksaan pekak beralih \\ \hline
	shifting dullness & Pemeriksaan undulasi & fluid thrill & Pemeriksaan colok dubur \\ \hline
	digital rectal examination & Palpasi sacrum & Inspeksi sarung tangan pasca colok dubur & Persiapan tinja \\ \hline
	pemeriksaan tinja & Pemasangan pipa nasogastrik & NGT & Endoskopi \\ \hline
	Nasogastric suction & Mengganti kantong pada kolostomi & Enema & Anal swab \\ \hline
	Identifikasi parasit & Pemeriksaan feses & Endoskopi lambung & Proktoskopi \\ \hline
	Biopsi hepar & Pengambilan cairan asites & Pemeriksaan bimanual ginjal & Pemeriksaan nyeri ketok ginjal \\ \hline
	Perkusi kandung kemih & Palpasi prostat & Refleks bulbokavernosus & Swab uretra \\ \hline
	Persiapan dan pemeriksaan sedimen urine & Uroflowmetry & Micturating cystigraphy & Pemeriksaan urodinamik \\ \hline
	Metode dip slide & Permintaan pemeriksaan BNOIVP & Interpretasi BNOIVP & Pemasangan kateter uretra \\ \hline
	Clean intermitten chateterization & Neurogenic bladder & Sirkumsisi & Pungsi suprapubik \\ \hline
	Dialisis ginjal & Inspeksi penis & Inspeksi skrotum & Palpasi penis \\ \hline
	Palpasi testis & Palpasi duktus spermatik epididimis & Transluminasi skrotum & Pemeriksaan fisikumum termasuk pemeriksaan payudara \\ \hline
	Inspeksi dan palpasi genitalia eksterna & Pemeriksaan spekulum: inspeksi vagina dan serviks & Pemeriksaan bimanual & Pemeriksaan rektal \\ \hline
	Pemeriksaan combined recto-vaginal & Melakukan swab vagina & Duh genital & Melakukan PapâA ? Zs smear ´ \\ \hline
	Pemeriksaan IVA & Kolposkopi & Pemeriksaan kehamilan USG periabdominal & Kuretase \\ \hline
	Laparoskopi diagnostik & Penilaian hasil pemeriksaan semen & Kurva temperatur basal & Pemeriksaan mukus serviks \\ \hline
	Tes fern & Uji pasca koitus & Histerosalpingografi & HSG \\ \hline
	Peniupan tuba Fallopi & Inseminasi artifisia & Melatih pemeriksaan payudara sendiri & Insersi pessarium \\ \hline
	Electro or crycoagulation cervix & Laparoskopi & terapeutik & Insisi abses Bartholini \\ \hline
	Konseling kontrasepsi & Insersi IUD & ekstraksi IUD & Laparoskopi \\ \hline
	sterilisasi & Insersi implant & ekstraksi implant & Kontrasepsi injeksi \\ \hline
	Penanganan komplikasi KB & IUD & pil & suntik \\ \hline
	implant & Identifikasi kehamilan risiko tinggi & Konseling prakonsepsi & Pelayanan perawatan antenatal \\ \hline
	Inspeksi abdomen wanita hamil & Palpasi tinggi fundus & Palpasi manuver Leopold & Palpasi penilaian posisi dari luar \\ \hline
	Mengukur denyut jantung janin & Pemeriksaan dalam pada kehamilan muda & Pemeriksaan pelvimetri klinis & Tes kehamilan \\ \hline
	CTG & Permintaan pemeriksaan USG obsgin & Pemeriksaan USG obsgin & skrining obstetri \\ \hline
	Amniosentesis & Chorionic villus sampling & Pemeriksaan obstetri & penilaian serviks \\ \hline
	penilaian dilatasi & penilaian membran & penilaian presentasi janin dan penurunan & Pemecahan membran ketuban sesaat sebelum melahirkan \\ \hline
	Insersi kateter untuk tekanan intrauterus & Anestesi lokal diperineum & Anestesi pudendal & Anestesi epidural \\ \hline
	Episiotomi & Resusitasi bayi baru lahir & Menilai skor Apgar & Pemeriksaan fisik bayi baru lahir \\ \hline
	Postpartum & pemeriksaan tinggi fundus & mengukur kehilangan darah sesudah melahirkan & Menjahit luka episiotomi serta laserasi \\ \hline
	Menjahit luka episiotomi & Insiasi menyusui dini & IMD & Induksi kimiawi persalinan \\ \hline
	Menolong persalinan dengan presentasi bokong & breech presentation & Pengambilan darah fetus & Operasi Caesar \\ \hline
	Caesarean section & Pengambilan plasenta secara manual & Ekstraksi vakum rendah & Pertolongan distosia bahu \\ \hline
	Kompresi bimanual & Menilai lochia & Palpasi posisi fundus & Mengajarkan hygiene \\ \hline
	Konseling kontrasepsi & KB pascasalin & Perawatan luka episiotomi & Perawatan luka operasi caesar \\ \hline
	Penilaian status gizi & pemeriksaan antropometri & Penilaian kelenjar tiroid & Pengaturan diet \\ \hline
	Penatalaksanaan diabetes melitus tanpa komplikasi & Pemberian insulin & Pemeriksaan gula darah & Pemeriksaan glukosa urine \\ \hline
	Palpasi kelenjar limfe & Persiapan dan pemeriksaan hitung jenis leukosit & Pemeriksaan darah rutin & Pemeriksaan profil pembekuan \\ \hline
	Pemeriksaan Laju endap darah & Permintaan pemeriksaan hematologi berdasarkan indikasi & Permintaan pemeriksaan imunologi berdasarkan indikasi & Skin test \\ \hline
	pemberiaan obat injeksi & Pemeriksaan golongan darah & inkompatibilitas & Penentuan indikasi dan jenis transfusi \\ \hline
	Inspeksi gait & Inspeksi tulang belakang saat berbaring & Inspeksi tulang belakang saat bergerak & Inspeksi tonus otot ekstremitas \\ \hline
	Inspeksi sendi ekstremitas & Inspeksi postur tulang belakang dan pelvis & Inspeksi posisi skapula & Inspeksi fleksi dan ekstensi punggung \\ \hline
	Penilaian fleksi lumbal & penilaian fleksi dan ekstensi & Menilai atrofi otot & menilai ligamen krusiatus \\ \hline
	menilai ligamen kolateral & Penilaian meniskus & inspeksi postur dan bentuk & penilaian fleksi dorsal \\ \hline
	Palpation for tenderness & Palpasi untuk mendeteksi nyeri diakibatkan tekanan vertikal & Palpasi tendon dan sendi & Palpasi tulang belakang \\ \hline
	sendi sakro-iliaka & Percussion for tenderness & Penilaian range ofmotion & Menetapkan ROM kepala \\ \hline
	Tes fungsi otot dan sendi bahu & Tes fungsi sendi pergelangan tangan & Pengukuran panjang ekstremitas bawah & Reposisi fraktur tertutup \\ \hline
	Stabilisasi fraktur & Reduksi dislokasi & Melakukan dressing & Nail bed cauterization \\ \hline
	Aspirasi sendi & Mengobati ulkus tungkai & Removal of splinter & Inspeksi kulit \\ \hline
	Inspeksi membran mukosa & Inspeksi daerah perianal & Inspeksi kuku & Inspeksi rambut danskalp \\ \hline
	Palpasi kulit & Pemeriksaan dermografisme & Penyiapan dan penilaian sediaan kalium hidroksida & Penyiapan dan penilaian sediaan metilen biru \\ \hline
	Penyiapan dan penilaian sediaan Gram & Biopsi plong & Uji tempel & patch test \\ \hline
	Uji tusuk & prick test & Pemeriksaan dengan sinar UVA & Pemilihan obat topikal \\ \hline
	Insisi dan drainase abses & Eksisi tumor jinak kulit & Ekstraksi komedo & Perawatan luka \\ \hline
	Kompres & Bebat kompresi pada vena varikosum & Rozerplasty kuku & Pencarian kontak \\ \hline
	Anamnesis dari pihak ketiga & Menelusuri riwayat makan & Anamnesis anak yang lebih tua & Berbicara dengan orang tua yang cemas \\ \hline
	Pemeriksaan fisik umum & Penilaian keadaan umum & Pengamatan malformasi kongenital & Palpasi fontanella \\ \hline
	Respons moro & Refleks menggenggam palmar & Refleks mengisap & Refleks melangkah \\ \hline
	Vertical suspension positioning & Asymmetric tonicneck reflex & Refleks anus & Penilaian panggul \\ \hline
	Penilaian pertumbuhan dan perkembangan anak & penilaian motorik halus & penilaian motorik kasar & penilaian psikososial \\ \hline
	penilaian bahasa & Pengukuran antropometri & Pengukuran suhu & Tes fungsi paru \\ \hline
	Ultrasound kranial & Pungsi lumbal & Ekokardiografi & Tes Rumple Leed \\ \hline
	Tatalaksana BBLR & KMC incubator & Tatalaksana bayibaru lahir dengan infeksi & Peresepan makanan untuk bayi yang mudah dipahami ibu \\ \hline
	Tatalaksana gizi buruk & Pungsi vena padaanak & Insersi kanula (venaperifer) pada anak & Insersi kanula (venasentral) pada anak \\ \hline
	Intubasi pada anak & Pemasangan pipa orofaring & Kateterisasi jantung & Vena seksi \\ \hline
	Kanulasi intraoseus & Tatalaksana anak dengan tersedak & Tatalaksana jalan nafas & pemberian oksigen \\ \hline
	Tatalaksana anak dengan kondisi tidak sadar & Tatalaksana pemberian infus pada anak syok & Tatalaksana pemberian cairan glukosa IV & Tatalaksana dehidrasi berat pada kegawatdaruratan setelah penatalaksanaan syok \\ \hline
	Penilaian keadaan umum & Penilaian antropologi & Penilaian kesadaran & Punksi vena \\ \hline
	Punksi arteri & Finger prick & pemeriksaan X-ray & Pemeriksaan skintigrafi \\ \hline
	Ekokardiografi & Pemeriksaan patologi hasil biopsi & Artrografi & Ultrasound skrining abdomen \\ \hline
	Biopsi & Menasehati pasien tentang gaya hidup & Anestesi infiltrasi & Blok saraf lokal \\ \hline
	Jahit luka & Pengambilan benang jahitan & Menggunakan anestesi topikal & Pemberian analgesik \\ \hline
	Vena seksi & Bantuan hidup dasar & Ventilasi masker & Intubasi \\ \hline
	Transpor pasien & transport of casualty & Manuver Heimlich & Resusitasi cairan \\ \hline
	Pemeriksaan turgor kulit & Pemeriksaan selaput dara & Pemeriksaan anus & Deskripsi luka \\ \hline
	Pemeriksaan derajat luka & Pemeriksaan label mayat & Pemeriksaan baju mayat & Pemeriksaan lebam mayat \\ \hline
	Pemeriksaan kaku mayat & Pemeriksaan tanda-tanda asfiksia & Pemeriksaan gigi mayat & Pemeriksaan lubang-lubang pada tubuh \\ \hline
	Pemeriksaan korban trauma dan deskripsi luka & Pemeriksaan patah tulang & Pemeriksaan tanda tenggelam & Pemeriksaan rongga kepala \\ \hline
	Pemeriksaan rongga dada & Pemeriksaan rongga abdomen & Pemeriksaan sistem urogenital & Pemeriksaan saluran luka \\ \hline
	Pemeriksaan uji apung paru & Pemeriksaan getah paru & Vaginal swab & Buccal swab \\ \hline
	Pengambilan darah & Pengambilan urine & Pengambilan muntahan atau isi lambung & Pengambilan jaringan \\ \hline
	Pengambilan sampel tulang & Pengambilan sampel gigi & Pengumpulan danpengemasan barangbukti & Pemeriksaan bercakdarah \\ \hline
	Pemeriksaan cairan mani & Pemeriksaan sperma & Histopatologi forensik & Fotografo forensik \\ \hline
\end{longtable}